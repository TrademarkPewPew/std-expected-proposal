\documentclass[xcolor=dvipsnames]{beamer}

\usepackage[utf8]{inputenc}
\usepackage[english]{babel}
\usepackage{url}
\usepackage{lmodern}
\usepackage{listings}
\usepackage{graphicx}
\usepackage{xcolor}
\usepackage{textcomp} 
\usepackage{hyperref}
\usepackage{subcaption}
\usepackage[T1]{fontenc}

\usepackage{color}
 
\definecolor{dkgreen}{rgb}{0,0.6,0}
\definecolor{gray}{rgb}{0.5,0.5,0.5}
\definecolor{mauve}{rgb}{0.58,0,0.82}
 
\lstset{
  language=c++,                % the language of the code
  basicstyle=\footnotesize,       % the size of the fonts that are used for the code
  numbers=left,                   % where to put the line-numbers
  numberstyle=\tiny\color{gray},  % the style that is used for the line-numbers
  stepnumber=1,                   % the step between two line-numbers. If it's 1, each line 
                                  % will be numbered
  numbersep=5pt,                  % how far the line-numbers are from the code
  backgroundcolor=\color{white},      % choose the background color. You must add \usepackage{color}
  showtabs=false,                 % show tabs within strings adding particular underscores
  frame=single,                   % adds a frame around the code
  rulecolor=\color{black},        % if not set, the frame-color may be changed on line-breaks within not-black text (e.g. comments (green here))
  tabsize=2,                      % sets default tabsize to 2 spaces
  captionpos=b,                   % sets the caption-position to bottom
  breaklines=true,                % sets automatic line breaking
  breakatwhitespace=false,        % sets if automatic breaks should only happen at whitespace
  title=\lstname,                   % show the filename of files included with \lstinputlisting;
                                  % also try caption instead of title
  keywordstyle=\color{blue},          % keyword style
  commentstyle=\color{dkgreen},       % comment style
  stringstyle=\color{mauve},         % string literal style
  morekeywords={expect, await, constexpr, explicit}
}

\newcommand{\cpp}[1]{\lstinline{#1}}

\setbeamertemplate{footline}{\hspace*{.5cm}\scriptsize{\insertauthor\hspace*{50pt} \hfill\insertframenumber\hspace*{.5cm}}}

\setbeamertemplate{section in toc}{%
\textcolor{MidnightBlue}{$\blacktriangleright$ \inserttocsection}
}

\usecolortheme{seahorse}
\usecolortheme{rose}
\useoutertheme{infolines}

\usecolortheme[named=SkyBlue]{structure}
\setbeamercolor{block title}{fg=MidnightBlue}

\AtBeginSection[]
{
  \begin{frame}<beamer>{Content}
    \tableofcontents[currentsection,hideothersubsections]
  \end{frame}
} 

\title{\textsc{Monads}: Alternative designs }
\subtitle{C++Now 2014}
\author[\textsc{Vicente Botet}]{Vicente \textsc{Botet Escriba}\texttt{vicente.botet@wanadoo.fr}}
\institute[Alcatel-Lucent]{Alcatel-Lucent International-Lannion}
\date[]{2014, May 16}

\begin{document}
\maketitle

\section{Introduction}
%%%%%%%%%%

\subsection{About me}
%%%%%%%%%%%

\begin{frame}
\frametitle{About Vicente Botet Escriba}

% \begin{figure}[p]
%   \centering
  % \begin{subfigure}[b]{0.3\textwidth}
  %   \includegraphics[scale=0.3]{images/vicente_botet.png}
  % \end{subfigure}
  % \qquad \qquad \quad
  % \begin{subfigure}[b]{0.3\textwidth}
  %   \includegraphics[scale=0.3]{images/alu.jpg}
  % \end{subfigure}
% \end{figure}

\begin{itemize}
\item Spanish.
\item Work for Alcatel-Lucent International in Lannion - France.
\item Interest in software engineering, language design, concurrency and STM.
\end{itemize}

\begin{block}{Open-source involvement}
\begin{itemize}
\item \textbf{Boost C++ library} Co-author and maintainer of Boost.Ratio, Boost.Chrono, Boost.Thread and some utilities in Boost.Utility.
\item \textbf{Toward Boost C++ library} author of a lot of unfinished prototypes.
\end{itemize}
\end{block}
\end{frame}

\subsection{About Boost.Monads}
%%%%%%%%%%%%
\begin{frame}
\frametitle{About Boost.Monads}

\begin{itemize}
\item After seen that \cpp{expected<E,T>}, \cpp{optional<T>} and \cpp{future<T>} all could share the same Monad interface.
\item we are looking for this common interface.
\end{itemize}

\begin{itemize}
\item Git repository : https://github.com/ptal/Boost.Expected
\end{itemize}
\end{frame}

\begin{frame}[fragile]
\frametitle{Motivation}

Problem:

\begin{itemize}
  \item We want to define some algorithms that work for \cpp{expected<E,T>}, \cpp{optional<T>} and \cpp{future<T>} between others.
  \item While these share a common underlying semantic, the interfaces are different.
  \item We need to define a common interface. 
\end{itemize}

\end{frame}
%%%%%%%%%%%%%%%%%%%%%%%%%%%%
\section{Concepts design}
%%%%%%%%%%%%%%%%%%%%%%%%%%%%
\begin{frame}[fragile]
\frametitle{Model to Concept Mapping}

Should the mapping be automatic, that is, have a default mapping that is applied implicitly if the type conforms to some syntactical requirements? 

\begin{itemize}
  \item As concepts are more than syntax the explicit mapping seems mandatory. 
  \item However this will be cumbersome, and the C++ standard library is based on syntax requirements.
\end{itemize}

Could several types share the same mapping?

\begin{itemize}
  \item Several types could already be a model of another Concept that can be used to implement the mapping. 
\end{itemize}

Could several mappings be applied to the same type?

\begin{itemize}
  \item A type can be seen in several ways as a model of given concept. 
\end{itemize}


\end{frame}
%%%%%%%%%%%%%%%%%%%%%%%%%%%%
\begin{frame}[fragile]
\frametitle{Alternative designs}

\begin{itemize}
  \item Direct: the model must provide the specific operations to conform to the Concept, E.g. the expression \cpp{*e} is well formed.
  \item Indirect: The concept is defined in terms of non-member functions. The developer of the model defines these non member functions for the specific model. E.g. the Concept requires that expression \cpp{deref(e)} is well formed and the Model mapping states how this operation is implemented using the Model.
  \item Mapper: All the features are mapped from the concept interface to the model interface. 
\end{itemize}
\end{frame}
%%%%%%%%%%%%%%%%%%%%%%%%%%%%
\begin{frame}[fragile]
\frametitle{Advantages/Liabilities}

\begin{itemize}
  \item The direct alternative would need to change the interface of the Model to conform to the Concept, or to add a thin wrapper that adapts the interface and force the user to use the adaptor.
  \item The indirect and mapper alternatives are more open but both introduce non-member functions. 
  \item The indirect approach could incur on violations of the ODR.  
  \item In the mapper approach all the functions would need a mapper as template parameter as if the function is executed under this mapper.
  \item Only the mapper approach allows to have multiple mappings of a model to a concept.  
  
\end{itemize}

\end{frame}
%%%%%%%%%%%%%%%%%%%%%%%%%%%%%%%%%%%%
\begin{frame}[fragile]
\frametitle{Concept usage comparison}

\begin{lstlisting}
template < class PV> // Direct
constexpr bool equal( const PV& x, const PV& y ) 
{
  return bool(x) != bool(y) ? false 
    : ( bool(x) ? *x == *y : true );
}
template <class PV>  // Indirect
constexpr bool equal( const PV& x, const PV& y ) 
{
  return has_value(x) != has_value(y) ? false 
    : ( has_value(x) ? deref(x) == deref<M>(y) : true );
}
template <class M, class PV> // Mapper
constexpr bool equal( const PV& x, const PV& y ) 
{
  return has_value<M>(x) != has_value<M>(y) ? false 
   : ( has_value<M>(x) ? deref<M>(x) == deref<M>(y) : true );
}
\end{lstlisting}

\end{frame}

%%%%%%%%%%%%%%%%%%%%%%%%%%%%
\section{Monadic Concepts}
%%%%%%%%%%%%%%%%%%%%%%%%%%%%
\subsection{Rebindable}
%%%%%%%%%%%%

\begin{frame}[fragile]
\frametitle{Rebindable features}

Rebindable is a basic concept that provides a way to

\begin{itemize}
  \item \cpp{value_type<M>} : obtain the type of the underlying value, 
  \item \cpp{type_constructor<M>} : obtain the type constructor. 
  \item \cpp{rebind<M,T>} : build another type substituting the underlying value type. 
\end{itemize}

satisfying 

\begin{lstlisting}
is_same<apply<type_constructor<M>, value_type<M>>, M>::value
is_same<type_constructor<apply<TC, U>>, TC>::value
is_same<rebind<M, value_type<M>>, M>::value
is_same<rebind<M, U>, apply<type_constructor<M>,U>>::value
\end{lstlisting}

\end{frame}
%%%%%%%%%%%%%%%%%%%%%%%%%%%%%%%%%%%%
\begin{frame}[fragile]
\frametitle{Rebindable features}

Where

\begin{lstlisting}
template<class F, class... Args>
using apply = typename F::template type<Args...>;
\end{lstlisting}

Examples:
\begin{itemize}
  \item pointers:  \cpp{T*}, 
  \item smart pointers: \cpp{shared_ptr<T>}, 
  \item containers, \cpp{vector<T>},  \cpp{array<T,N>},  \cpp{T[N]} 
  \item wrappers: \cpp{optional<T>}, \cpp{expected<E, T>}, \cpp{future<T>},
  \item other: \cpp{allocator<T>}   
\end{itemize}

\end{frame}
%%%%%%%%%%%%%%%%%%%%%%%%%%%%
\subsection{PossiblyValued}
%%%%%%%%%%%%%%%%%%%%%%%%%%%%%%%%%%%%
\begin{frame}[fragile]
\frametitle{PossiblyValued features}

PossiblyValued is a refinement of a Rebindable  concept that allows to

\begin{itemize}
  \item \cpp{has_value(m)}: check if a value is present.
  \item \cpp{deref(m)}: obtain a reference to the value. (defined only if \cpp{has_value}).
  \item \cpp{value(m)}: get a reference to the stored value or throw an exception.
  
  \item \cpp{error_type<M>}: obtain the associated error type.
  \item \cpp{error(m)}: get a reference to the stored error (defined only if \cpp{has_value} is false).
  
  \item \cpp{errored_type<M>}: obtain the associated errored type, which is implicitly convertible to the model.
  \item \cpp{make_errored(m)}: get a value storing the stored error that is convertible to the model. 
 
\end{itemize}
\end{frame}
%%%%%%%%%%%%%%%%%%%%%%%%%%%%%%%%%%%%
\begin{frame}[fragile]
\frametitle{PossiblyValued algorithms}

\begin{lstlisting}
template <class PV> 
// requires PossiblyValued<PV> 
// && EqualityComparable<value_type<PV>>
constexpr bool equal( const PV& x, const PV& y ) {
  return has_value<M>(x) != has_value<M>(y) ? false 
       : ( has_value<M>(x) ? deref<M>(x) == deref<M>(y) : true );
}
\end{lstlisting}

\begin{lstlisting}
template <class PV, class U> 
// requires PossiblyValued<PV> 
// && Convertible<value_type<PV>,U>
constexpr value_type<PV> value_or( const PV& x, U&& y ) {
  return has_value<M>(x) ? deref<M>(x) : value_type<M,PV>(y); 
}
\end{lstlisting}

\end{frame}
%%%%%%%%%%%%%%%%%%%%%%%%%%%%%%%%%%%%
\subsection{Functor}
%%%%%%%%%%%%%%%%%%%%%%%%%%%%%%%%%%%%
\begin{frame}[fragile]
\frametitle{C++ 'Functor'}

\cpp{Functor} is a refinement of Rebindable for type wrappers that can be mapped over.  It provides a

\begin{itemize}
  \item \cpp{fmap(f,m)}: traverse a functor applying a function.
\end{itemize}
        
satisfying

\begin{lstlisting}
bind(fmap,id,_1)  ==  bind(id,_1)
bind(fmap, compose(f, g), _1)  ==  
        compose(bind(fmap, f,_1), bind(fmap, g, _1)
\end{lstlisting}

\end{frame}
%%%%%%%%%%%%%%%%%%%%%%%%%%%%
\begin{frame}[fragile]
\frametitle{\cpp{Functor}: Overloads of \cpp{fmap}}

\begin{lstlisting}
  template <class F, class M>
  auto fmap(F&& f, M&& m)
       -> decltype( m.fmap(forward<F>(f)) )   {
    return m.fmap(forward<F>(f));
  }
  // overload for optional<T>
  template <class F, class T>
  auto fmap(F&& f, optional<T>&& m) 
       -> decltype(make_optional(f(*m)))    {
    if (m) return make_optional(f(*m));
    else return nullopt;
  }
\end{lstlisting}

\begin{lstlisting}
optional<int> f1;
auto f2 = fmap(fct, f1); 
\end{lstlisting}

\end{frame}
%%%%%%%%%%%%%%%%%%%%%%%%%%%%
\begin{frame}[fragile]
\frametitle{\cpp{Functor}: Chain syntax for \cpp{fmap}}

\begin{lstlisting}
// very nice chain syntax. 
auto f2 = f1.fmap(fct1).fmap(fct2); 

// ugly functional syntax
auto f2 = fmap(fct2, fmap(fct1, f1));  

// nice chain syntax. 
optional<int> f2 = as_functor(f1).fmap(fct1).fmap(fct2); 
auto f2 = as_functor(f1).fmap(fct1).fmap(fct2).get();

// nice operator syntax.  
optional<int> f2 = as_functor(f1) >> fct1 >> fct2;  
\end{lstlisting}

\end{frame}
%%%%%%%%%%%%%%%%%%%%%%%%%%%%%%%%%%%%
\begin{frame}[fragile]
\frametitle{C++ Functor - Specialization for PossiblyValued}
       
Specialization for models of \cpp{PossiblyValued}       

\begin{lstlisting}
template <class F, class M>
// requires PossiblyValued<M> 
// && Invokable<F, value_type<M>>
rebind<M, result_of_t<F(value_type<M>)> fmap(F&& f, M&& m) {
  return has_value( forward<M>(m))
  ? forward<F>(f)( deref( forward<M>(m) ) )
  : make_errored(forward<M>(m) ) );
}
\end{lstlisting}
         
\end{frame}
%%%%%%%%%%%%%%%%%%%%%%%%%%%%%%%%%%%%
\subsection{Monad}
%%%%%%%%%%%%%%%%%%%%%%%%%%%%%%%%%%%%
\begin{frame}[fragile]
\frametitle{Monad}

\cpp{Monad} is a refinement of \cpp{Functor} concept  providing a way to:

\begin{itemize}
  \item \cpp{make<M>(v)}: build a monad instance from the underlying value type.
  \item \cpp{mbind(m, f)} : binds a function that will be called only if the action associated to the Monad succeeds.
  \item \cpp{mdo(m1, m2)} : sequentially compose two actions, discarding any value produced by the first.
\end{itemize}

satisfying

\begin{lstlisting}
mbind(make<M>(x), f)  ==  f(x)
mbind(m, [](auto x) { return make<M>(x);} )  ==  m
mbind(m, [](auto x) { return mbind(f(x), g); }  ==  
           mbind(mbind(m, f), g)

fmap(f, m)  ==  
  mbind(m, compose([](auto x) { return make<M>(x);}, f))
\end{lstlisting}
\end{frame}
%%%%%%%%%%%%%%%%%%%%%%%%%%%%%%%%%%%%
\begin{frame}[fragile]
\frametitle{Non-uniform type factories}

\begin{lstlisting}
auto e1 = make_expected(2); 
auto e2 = expected<>::make(2); 
auto e2 = expected<errc>::make(2);

auto o = make_optional(2); 
auto f = make_ready_future(2); 
\end{lstlisting}

\begin{itemize}
  \item Non-uniform syntax -> no generic algorithms 
\end{itemize}

\end{frame}
%%%%%%%%%%%%%%%%%%%%%%%%%%%%%%%%%%%%
\begin{frame}[fragile]
\frametitle{Uniform type factories - type constructor}

\begin{itemize}
  \item Type constructor 
\end{itemize}

\begin{lstlisting}
auto e2 = make<expected<>>(2); 
auto e2 = make<expected<errc>>(2);
auto o = make<optional<>>(2); 
auto f = make<future<>>(2); 
\end{lstlisting}
\end{frame}
%%%%%%%%%%%%%%%%%%%%%%%%%%%%%%%%%%%%
\begin{frame}[fragile]
\frametitle{\cpp{expected<E>} as a type constructor}

\begin{lstlisting}
auto e = make<expected<errc>>(2); 
// e has type expected<errc, int>
auto e = make<expected<>>(2); 
// e has type expected<exception_ptr, int>
\end{lstlisting}

\begin{lstlisting}
template<typename TC, class T>
constexpr apply<TC,decay_t<T>> make(T&& v)  
{
  return apply<TC,decay_t<T>>(std::forward(v));
}
\end{lstlisting}

\end{frame}
%%%%%%%%%%%%%%%%%%%%%%%%%%%%
\begin{frame}[fragile]
\frametitle{Uniform type factories - Lifting }

\begin{itemize}
  \item Lifting variadic template class to create a type constructor 
\end{itemize}

\begin{lstlisting}
template <template <class ...> class F, class... Args>
struct lift {
  template<class... Args2> using type = F<Args..., Args2...>;
};
\end{lstlisting} 

\begin{lstlisting}
auto e2 = make<lift<expected, exception_ptr>>(2); 
auto e2 = make<lift<expected, errc>>(2);
auto o = make<lift<optional>>(2); 
auto f = make<lift<future>>(2); 
\end{lstlisting}

\end{frame}
%%%%%%%%%%%%%%%%%%%%%%%%%%%%
\begin{frame}[fragile]
\frametitle{\cpp{lift<future>} as a type constructor?}

\begin{itemize}
  \item However, \cpp{future<T>} is not explicitly constructible from \cpp{T}.
\end{itemize}

\begin{lstlisting}
auto f = make_ready_future(2); 
auto f = make<lift<future>>(2); 
\end{lstlisting}

\begin{itemize}
  \item Either \cpp{future<T>} is constructible from \cpp{T}.
  \item Either we add an indirection that maps the model to the concept.
\end{itemize}

\end{frame}
%%%%%%%%%%%%%%%%%%%%%%%%%%%%
\begin{frame}[fragile]
\frametitle{\cpp{lift<future>} as a type constructor}

\begin{lstlisting}
template<class TC, class T>
apply<TC,T>  make(T&& v)  {
    TC* ptr=0; 
    return make(std::forward<T>(v), ptr);
}
template<class TC, class T>
apply<TC,T>  make(T&& v, TC*)  {
    return apply<TC,T>(std::forward<T>(v));
}
template<class T>
future<T>  make(T&& v, lift<future>*)  {
    return make_ready_future(std::forward<T>(v));
}
\end{lstlisting}
\begin{lstlisting}
auto f = make<lift<future>>(2); 
\end{lstlisting}

\end{frame}
%%%%%%%%%%%%%%%%%%%%%%%%%%%%
\begin{frame}[fragile]
\frametitle{\cpp{future<T>} mapping for \cpp{mbind}}

\begin{lstlisting}
struct future_m {
  // ...
  template <class M, class F>
  static auto mbind(M&& m, F&& f) {
    return m.then(if_valued(f))
  }
};
\end{lstlisting}

where \cpp{if_valued} is an adaptor that calls \cpp{f} when the parameter has a value.  

\begin{lstlisting}
auto f2 = mbind<future_m>(f1, fct); 
\end{lstlisting}

\end{frame}
%%%%%%%%%%%%%%%%%%%%%%%%%%%%%%%%%%%%
\begin{frame}[fragile]
\frametitle{\cpp{mbind}  specialization for PossiblyValued}
       
\begin{lstlisting}
template <class M, class F>
// requires PossiblyValued<M> && Invokable<F, value_type<M>>
result_of_t<F(value_type<M>)> mbind(M&& m, F&& f) {
  return has_value( forward<M>(m))
  ? forward<F>(f)( deref( forward<M>(m) ) )
  : make_errored( forward<M>(m) ) );
}
\end{lstlisting}
         
\end{frame}
%%%%%%%%%%%%%%%%%%%%%%%%%%%%%%%%%%%%
\subsection{MonadError}
%%%%%%%%%%%%%%%%%%%%%%%%%%%%%%%%%%%%
\begin{frame}[fragile]
\frametitle{MonadError}

A MonadError is a Monad which can store errors of different types.

\begin{itemize}
  \item \cpp{make_error<M>(e)}: signal an error on a monadic function.
  \item \cpp{catch_error(m, f)}: possibly recover from previous errors and return to normal execution.
\end{itemize}

Should satisfy the laws:

\begin{lstlisting}
  fmap(make_error<M>(e),  f) == make_error<M>(e)
  mbind(make_error<M>(e),  f) == make_error<M>(e)
  mdo(make_error<M>(e),  f) == make_error<M>(e)
  catch_error(make<M>(v),  f) == make<M>(v)
\end{lstlisting}
\end{frame}

\subsection{MonadException}
%%%%%%%%%%%%%%%%%%%%%%%%%%%%%%%%%%%%
\begin{frame}[fragile]
\frametitle{MonadException}

A MonadException is a MonadError which can store errors of different types.

\begin{itemize}
  \item \cpp{make_exception<M>(e)}: signal an exception on a monadic function.
  \item \cpp{has_exception<E>(m)} : check if the stored exception is of the type parameter \cpp{E}.
  \item \cpp{catch_exception<E>(m, f)}: possibly recover from previous exception and return to normal execution.
\end{itemize}

Examples
\begin{itemize}
  \item \cpp{expected<exception_ptr,T>} 
  \item \cpp{expected<any,T>} 
  \item \cpp{expected<variant<E1, ..., En>,T>} 
  \item \cpp{future<T>} 
  \item \cpp{shared_future<T>} 
\end{itemize}
\end{frame}
%%%%%%%%%%%%%%%%%%%%%%%%%%%%%%%%%%%%
\section{Open Points, Future Work and Conclusions}
%%%%%%%%%%%%%%%%%%%%%%%%%%%%%%%%%%%%
\begin{frame}
\frametitle{Open Points ...}

\begin{itemize}
  \item Which design should be proposed?
\end{itemize}

\end{frame}
%%%%%%%%%%%%%%%%%%%%%%%%%%%%%%%%%%%%
\begin{frame}
\frametitle{Future Work ...}

\begin{itemize}
  \item Define a common visitation interface for \cpp{any}, \cpp{variant<E1, ..., En>}, \cpp{exception_ptr}.
  \item Define an Error concept and an Exception concept that would make \cpp{expected<Error,T>} more generic.
\end{itemize}

\end{frame}

\begin{frame}
\frametitle{Conclusions}
%%%%%%%%%%%%
\begin{itemize}
  \item Monadic functions are scary useful tools, 
  \item that allows to combine functions that return monads but,
  \item it would be much easier to use it with a specific language do-expression.
\end{itemize}

\begin{itemize}
  \item \cpp{expected<E, T>}, \cpp{future<T>} and \cpp{optional<T>} share a lot of things but have some differences.
  \item Defining a common interface for the functions that have the same behavior allows us to define generic algorithms on top of these concepts.
  \item Having a monadic common interface would be one step towards this goal.
\end{itemize}

\end{frame}

\begin{frame}
%%%%%%%%%%%%
\begin{center}
\Large{Thanks for your attention!}
\end{center}
\begin{center}
\Large{Questions?}
\end{center}
\end{frame}


%%%%%%%%%%%%%%%%%%%%%%%%%%%%
%%%%%%%%%%%%%%%%%%%%%%%%%%%%
%%%%%%%%%%%%%%%%%%%%%%%%%%%%



%%%%%%%%%%%%%%%%%%%%%%%%%%%%
\begin{frame}[fragile]
\frametitle{language-like style  - error propagation}
C++ language extension: Based on the Haskell do-expression. Something similar to the \cpp{await} expression for futures.  

\begin{lstlisting}
do_expression ::= do_initialization ':' do_expression_or_make_expression
do_expression_or_make_expression := 
    do_expression | make_expression
do_initialization := type var '<-' expression
\end{lstlisting}

\begin{itemize}
  \item Transformation semantics.
\end{itemize}

\begin{lstlisting}
[[do_expression]] =
  expression.mbind([&](type var) {
    return [[do_expression_or_make_expression]]   });
    
\end{lstlisting}

\end{frame}

%%%%%%%%%%%%%%%%%%%%%%%%%%%%
\begin{frame}[fragile]
\frametitle{\cpp{i/k + j/k}  - DO-macro}

\begin{lstlisting}
expected<exception_ptr,int> f2(int i, int j, int k) {
  return DO (
    ( s1, safe_divide(i, k) )
    ( s2, safe_divide(j, k) )
    s1 + s2 
  );
}
\end{lstlisting}

\end{frame}





%%%%%%%%%%%%%%%%%%%%%%%%%%%%
\begin{frame}[fragile]
\frametitle{\cpp{Functor}: Mapper for \cpp{fmap}}

\begin{lstlisting}
template <class Mapper, class F, class M>
auto fmap(F&& f, M&& m)
    -> decltype(Mapper::fmap(forward<F>(f), forward<M>(m)))
{
  return Mapper::fmap(forward<F>(f), forward<M>(m));
}
// functor mapper for optional
struct optional_m {
  // ...
  template <class F, class T>
  static auto fmap(F&& f, optional<T>&& m) 
       -> decltype(make_optional(f(*m)))    {
  {
    if (m) return make_optional(f(*m));
    else return nullopt;
  }
};
\end{lstlisting}

\begin{lstlisting}
auto f2 = fmap<optional_m>(fct, f1); 
\end{lstlisting}

\end{frame}
%%%%%%%%%%%%%%%%%%%%%%%%%%%%
\begin{frame}[fragile]
\frametitle{\cpp{Functor}: Uniform chain syntax for \cpp{fmap}}

\begin{lstlisting}
// very ugly functional syntax
auto f2 = fmap<optional_m>(fct2, fmap<mapper>(fct1, f1)); 

// nice chain syntax. 
optional<int> f2 = 
    as_functor<optional_m>(f1).fmap(fct1).fmap(fct2); 
auto f2 = 
    as_functor<optional_m>(f1).fmap(fct1).fmap(fct2).get();

// nice operator syntax.  
optional<int> f2 = 
    as_functor<optional_m>(f1) >> fct1 >> fct2;  
\end{lstlisting}

\end{frame}
%%%%%%%%%%%%%%%%%%%%%%%%%%%%
\begin{frame}[fragile]
\frametitle{\cpp{future_m} mapper as a type constructor}

\begin{lstlisting}
template<class Mapper, class T>
apply<Mapper,T>  make(T&& v )  
{
    return Mapper::template make(std::forward<T>(v));
}
\end{lstlisting}
\begin{lstlisting}
struct future_m {
  template <class T> using type = future<T>;
  template <class T> 
  static future<T>  make(T&& v )  {
    return make_ready_future(std::forward<T>(v));
}
\end{lstlisting}
\begin{lstlisting}
auto f = make<future_m>(2); 
\end{lstlisting}

\end{frame}
%%%%%%%%%%%%%%%%%%%%%%%%%%%%
\begin{frame}[fragile]
\frametitle{\cpp{Monad}: Uniform syntax for \cpp{mbind}}

\begin{lstlisting}
namespace monad {
  template <class Mapper, class M, class F>
  auto mbind(M&& m, F&& f)
    -> decltype(Mapper::mbind(forward<M>(m), forward<F>(f)))
  {
    return Mapper::mbind(forward<M>(m), forward<F>(f));
 }
}
\end{lstlisting}

\begin{lstlisting}
auto f2 = mbind<mapper>(f1, fct); 
\end{lstlisting}

\end{frame}
%%%%%%%%%%%%%%%%%%%%%%%%%%%%%%%%%%%%
\begin{frame}[fragile]
\frametitle{\cpp{catch_error} specialization for PossiblyValued}
       
\begin{lstlisting}
template <class M, class F>
// requires PossiblyValued<M> 
// && Invokable<F, error_type<M>>
M catch_error(M&& m, F&& f) {
  return ! has_value( forward<M>(m))
  ? forward<F>(f)( error( forward<M>(m) ) )
  : move(m);
}
\end{lstlisting}
         
\end{frame}
%%%%%%%%%%%%%%%%%%%%%%%%%%%%%%%%%%%%
\begin{frame}[fragile]
\frametitle{MonadException catch\_exception example}

\begin{lstlisting}
template <class M> apply<M, int> safe_divide(int i, int j)  
{
  if (j == 0) return make_exception<M>(DivideByZero());
  else return make<M>(i / j);
}
template <class M> apply<M, int> divide(int i, int j)  
{
  return catch_exception<NotDivisible>(
    safe_divide<M>(i,j), 
    [](auto& e)  { return make<M>(e.i / e.j);  }
  );
}
//...
  auto a = divide<expected<>>(1, 0);
\end{lstlisting}
\end{frame}
%%%%%%%%%%%%%%%%%%%%%%%%%%%%%%%%%%%%



\end{document}
