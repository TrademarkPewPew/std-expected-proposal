\documentclass[a4paper,10pt]{article}
\usepackage[american]{babel} % needed for iso dates
\usepackage[utf8]{inputenc}
\usepackage{url}
\usepackage{lmodern}
\usepackage{listings}
\usepackage{graphicx}
\usepackage{xcolor}
\usepackage[T1]{fontenc}
\usepackage{textcomp}
\usepackage{hyperref}
\usepackage{array}
\usepackage{underscore}
\usepackage{changepage}   % for the adjustwidth environment

\hypersetup{
  hidelinks
}

\newcommand{\todo}[1]{\emph{\textcolor{red}{TODO: #1}}}
\usepackage[top=2cm, bottom=2cm, left=2cm, right=2cm]{geometry}


\title{Proposal for the C++Now conference – Expected}
\author{Pierre T\textsc{albot} \and Vicente J. B\textsc{otet} E\textsc{scriba}}

\begin{document}

\maketitle
\todo{Several other name propositions:\\
The expected error monad.\\
Expected – An exceptional error monad.\\}
\section{Information}

\begin{itemize}
\item \textbf{Title} Expected — An exception-friendly error monad
\item \textbf{Type} Presentation.
\item \textbf{Optimum/minimum/maximum} 90/90/90 minutes.
\item \textbf{Level} Basic
\end{itemize}

\section{Abstract}

The Expected library has been first introduced by Alexandrescu during the C++ and Beyond conference in 2012. It is a new way to handle errors in C++ lying somewhere between the classic error-code returns and the exceptions. Expected is fully compatible with exception-throwing code and helps to design exception-free interface. This open the door to novel techniques enforcing error handling safety with a clean code using the monad theory (borrowed from functional language such as Haskell). This talk will give an overview of this library and will present typical examples using the Expected class.

\section{Authors}

\subsection{Pierre T\textsc{albot}}

Pierre Talbot is a student at the University of Pierre et Marie Curie in Paris. He discovered C++ and the open-source world in 2011 during a Google Summer of Code, thanks to Boost for being his first mentoring organization. Since he continued to work on open-source projects, in the Boost organization and more recently for the Battle for Wesnoth game.

\subsection{Vicente J. B\textsc{otet} E\textsc{scriba}}

Vicente J. Botet Escriba got a Master in Computer Science from University Complutense of Madrid- Spain in 1986. More than 25 years of experience on software engineering for fault tolerant systems and more than 15 years of C++ experience. His main research area is on Concurrent Systems and Parallel Computing. Since January 2008 contributes actively to the Boost community. Co-author and maintainer of Boost.Thread/Chrono/Ratio. He is software engineer at Alcatel-Lucent France where he focuses on adaptation of very large telecom systems to multi-core platforms.

\section{Presentation overview}


\end{document}