\documentclass[xcolor=dvipsnames]{beamer}

\usepackage[utf8]{inputenc}
\usepackage[english]{babel}
\usepackage{url}
\usepackage{lmodern}
\usepackage{listings}
\usepackage{graphicx}
\usepackage{xcolor}
\usepackage{textcomp} 
\usepackage{hyperref}
\usepackage{subcaption}
\usepackage[T1]{fontenc}

\usepackage{color}
 
\definecolor{dkgreen}{rgb}{0,0.6,0}
\definecolor{gray}{rgb}{0.5,0.5,0.5}
\definecolor{mauve}{rgb}{0.58,0,0.82}
 
\lstset{
  language=c++,                % the language of the code
  basicstyle=\footnotesize,       % the size of the fonts that are used for the code
  numbers=left,                   % where to put the line-numbers
  numberstyle=\tiny\color{gray},  % the style that is used for the line-numbers
  stepnumber=1,                   % the step between two line-numbers. If it's 1, each line 
                                  % will be numbered
  numbersep=5pt,                  % how far the line-numbers are from the code
  backgroundcolor=\color{white},      % choose the background color. You must add \usepackage{color}
  showtabs=false,                 % show tabs within strings adding particular underscores
  frame=single,                   % adds a frame around the code
  rulecolor=\color{black},        % if not set, the frame-color may be changed on line-breaks within not-black text (e.g. comments (green here))
  tabsize=2,                      % sets default tabsize to 2 spaces
  captionpos=b,                   % sets the caption-position to bottom
  breaklines=true,                % sets automatic line breaking
  breakatwhitespace=false,        % sets if automatic breaks should only happen at whitespace
  title=\lstname,                   % show the filename of files included with \lstinputlisting;
                                  % also try caption instead of title
  keywordstyle=\color{blue},          % keyword style
  commentstyle=\color{dkgreen},       % comment style
  stringstyle=\color{mauve},         % string literal style
  morekeywords={expect, await, constexpr, explicit}
}

\newcommand{\cpp}[1]{\lstinline{#1}}

\setbeamertemplate{footline}{\hspace*{.5cm}\scriptsize{\insertauthor\hspace*{50pt} \hfill\insertframenumber\hspace*{.5cm}}}

\setbeamertemplate{section in toc}{%
\textcolor{MidnightBlue}{$\blacktriangleright$ \inserttocsection}
}

\usecolortheme{seahorse}
\usecolortheme{rose}
\useoutertheme{infolines}

\usecolortheme[named=SkyBlue]{structure}
\setbeamercolor{block title}{fg=MidnightBlue}

\AtBeginSection[]
{
  \begin{frame}<beamer>{Content}
    \tableofcontents[currentsection,hideothersubsections]
  \end{frame}
} 

\title{\textsc{Expected}: Exception-friendly MonadError}
\subtitle{C++Now 2014}
\author[\textsc{Botet},\textsc{Talbot}]{Vicente \textsc{Botet Escriba}, Pierre \textsc{Talbot} \texttt{vicente.botet@wanadoo.fr/ptalbot@hyc.io}}
\institute[Alcatel-Lucent,UMPC]{Alcatel-Lucent International-Lannion/University of Pierre et Marie Curie-Paris (France)}
\date[]{2014, May 16}

\begin{document}
\maketitle

\section{Introduction}
%%%%%%%%%%

\subsection{About us}
%%%%%%%%%%%

%%%%%%%%%%%%%%%%%%%%%%%%%%%%
\begin{frame}
\frametitle{About Pierre Talbot}

\begin{itemize}
\item Belgian (French side).
\item Still a student, at the University Pierre et Marie Curie in Paris.
\item Interest in software engineering, language design and concurrency.
\end{itemize}

\begin{block}{Open-source involvement}
\begin{itemize}
\item \textbf{Boost C++ library} Working on Boost.Check and Boost.Expected
\item \textbf{Wesnoth} The add-on server (UMCD).
\end{itemize}
\end{block}
\end{frame}
%%%%%%%%%%%%%%%%%%%%%%%%%%%%
\begin{frame}
\frametitle{About Vicente Botet Escriba}

\begin{figure}[p]
  \centering
  \begin{subfigure}[b]{0.3\textwidth}
    \includegraphics[scale=0.3]{images/vicente_botet.png}
  \end{subfigure}
  \qquad \qquad \quad
  \begin{subfigure}[b]{0.3\textwidth}
    \includegraphics[scale=0.3]{images/alu.jpg}
  \end{subfigure}
\end{figure}

\begin{itemize}
\item Spanish.
\item Work for Alcatel-Lucent International in Lannion - France.
\item Interest in software engineering, language design, concurrency and STM.
\end{itemize}

\begin{block}{Open-source involvement}
\begin{itemize}
\item \textbf{Boost C++ library} Co-author and maintainer of Boost.Ratio, Boost.Chrono, Boost.Thread and some utilities in Boost.Utility.
\item \textbf{Toward Boost C++ library} author/co-autor of a lot of unfinished prototypes as Conversions, Stopwatch, Date, FixedPoint, Enum, Opaque, Bitfield, InterThreads, EndianExt, LUID and STM.
\end{itemize}
\end{block}
\end{frame}

\subsection{About Boost.Expected}
%%%%%%%%%%%%
\begin{frame}
\frametitle{About Boost.Expected}

\begin{itemize}
\item Based on the original idea of Andrei Alexandrescu (Systematic Error Handling in C++).
\item Interface based on the new \cpp{std::experimental::optional<T>} type,taking advantage of the new C++11/14 features.
\item Extended to conform to the MonadError concept.
\item The project born as a proposal the authors made for GSoC 2013.
\item We are currently working on a C++ standard proposal that we expect to be ready for Rapperswil.
\item Once the interface would be more stable we will make a portable implementation on C++98 compilers and propose it for review to Boost.
\end{itemize}

\begin{itemize}
\item Git repository : https://github.com/ptal/Boost.Expected
\item C++ draft proposal : http://www.hyc.io/boost/expected-proposal.pdf
\end{itemize}
\end{frame}
%%%%%%%%%%%%%%%%%%%%%%%%%%%%
\begin{frame}[fragile]
\frametitle{Overview}

\cpp{expected<E, T>} answer to the question "do you contain a value of type \cpp{T} or an error of type \cpp{E}?".

\begin{lstlisting}
template <class E, class T>
class expected {
  bool has_value_;
  union{ 
    T val_;
    E err_;
  };
};
\end{lstlisting}

\begin{itemize}
\item \cpp{expected<E,T> = T + unexpected_type<E>}
\item \cpp{T != unexpected_type<E>}
\item We want that \cpp{expected<E,T>} behaves as \cpp{T}.
\end{itemize}

\end{frame}
%%%%%%%%%%%%%%%%%%%%%%%%%%%%
\begin{frame}[fragile]
\frametitle{How to create an expected object}

\begin{itemize}
\item Valued.
\end{itemize}

\begin{lstlisting}
expected<exception_ptr, int> ei = make_expected(1); 
expected<exception_ptr, int> ej{2}; // alternative syntax
expected<exception_ptr, int> ek = ej; // ek has value 2

expected<exception_ptr,MoveOnly> el{in_place, "arg"};        
expected<exception_ptr,MoveOnly> el{expect, "arg"};        
  // calls MoveOnly{"arg"} in place
\end{lstlisting}

\end{frame}
%%%%%%%%%%%%%%%%%%%%%%%%%%%%
\begin{frame}[fragile]
\frametitle{How to create an unexpected object}

\begin{itemize}
\item Unexpected.
\end{itemize}

\begin{lstlisting}
expected<errc, int> ek = make_unexpected(errc::invalid);

expected<string,int> em{unexpect, "arg"};        
  // unexpected value, calls string{"arg"} in place

expected<string,int> e;  // e is unexpected...
ei = {}; // reset to unexpected
\end{lstlisting}

\end{frame}

%%%%%%%%%%%%%%%%%%%%%%%%%%%%
\begin{frame}[fragile]
\frametitle{Observing an expected object}

\begin{itemize}
\item Explicit conversion to bool.
\item \cpp{operator*()} narrow contract
\end{itemize}

\begin{lstlisting}
if (expected<exception_ptr, char> ch = readNextChar()) {
  process_char(*ch);
}
\end{lstlisting}

\begin{itemize}
\item \cpp{value()} wide contract
\end{itemize}

\begin{lstlisting}
expected<errc,int> ei = str2int(s);
try {
    process_int(ei.value());
} 
catch(bad_expected_access const&) {
    std::cout << "this was not a number";
}
\end{lstlisting}

\end{frame}


%%%%%%%%%%%%%%%%%%%%%%%%%%%%
\section{Expected Library by Example}
%%%%%%%%%%%%%%%%%%%%%%%%%%%%
\subsection{\cpp{safe_divide}}
%%%%%%%%%%%%%%%%%%%%%%%%%%%%
\begin{frame}[fragile]
\frametitle{safe\_divide exception-based style}

\begin{lstlisting}
struct DivideByZero: public std::exception {...};
\end{lstlisting}

\begin{lstlisting}
int safe_divide(int i, int j)
{
  if (j==0) throw DivideByZero();
  else return i / j;
}
\end{lstlisting}
\end{frame}
%%%%%%%%%%%%%%%%%%%%%%%%%%%%
\begin{frame}[fragile]
\frametitle{safe\_divide Expected-based style}

\begin{lstlisting}
Expected<int> safe_divide(int i, int j)
{
  if (j==0) return Expected<int>::from_error(DivideByZero());
  else return Expected<int>(i / j);
}
\end{lstlisting}
\begin{itemize}
  \item (3,4) redondant use of  \cpp{Expected<int>}. 
\end{itemize}

\end{frame}
%%%%%%%%%%%%%%%%%%%%%%%%%%%%
\begin{frame}[fragile]
\frametitle{safe\_divide expected-based style}

Using \cpp{boost::expected<exception_ptr, int>} to carry the error condition.

\begin{lstlisting}
expected<exception_ptr,int> safe_divide(int i, int j)
{
  if (j==0) return make_unexpected(DivideByZero());
  else return i / j;
}
\end{lstlisting}

\begin{itemize}
  \item (3) uses implicit conversion from \cpp{unexpected_type<E>} to \cpp{expected<E, T>}. 
  \item (4) uses implicit conversion from \cpp{T} to \cpp{expected<E, T>}.
\end{itemize}

\end{frame}
%%%%%%%%%%%%%%%%%%%%%%%%%%%%
\begin{frame}[fragile]
\frametitle{safe\_divide expected-style  - getting the value or an exception}

\begin{itemize}
  \item The caller obtains the stored value using the member \cpp{value()} function.
\end{itemize}

\begin{lstlisting}
  auto x = safe_divide(i,j).value(); // throw on error
\end{lstlisting}

\begin{itemize}
  \item The caller can also store the expected value and check if a value is stored using the bool conversion  function.
\end{itemize}

\begin{lstlisting}
  auto x = safe_divide(i,j); // won't throw
  if (! x) 
    int i = x.value(); // just "re"throw the stored exception
\end{lstlisting}

\end{frame}
%%%%%%%%%%%%%%%%%%%%%%%%%%%%
%%%%%%%%%%%%%%%%%%%%%%%%%%%%
\begin{frame}[fragile]
\frametitle{\cpp{i + j/k} expected-style  - getting the value when available}

\begin{itemize}
  \item When the user knows that there is a value it can use the narrowed contact \cpp{operator*()}.
\end{itemize}

\begin{lstlisting}
expected<exception_ptr,int> f1(int i, int j, int k)
{
  auto q = safe_divide(j, k)
  if(q) return i + *q;
  else return make_unexpected(q);
}
\end{lstlisting}

\end{frame}
\subsection{\cpp{i + j/k}}
%%%%%%%%%%%%%%%%%%%%%%%%%%%%
\begin{frame}[fragile]
\frametitle{\cpp{i + j/k}  - functional style}

\begin{lstlisting}
expected<exception_ptr,int> f1(int i, int j, int k)
{
  return safe_divide(j, k).mbind([i](int q){return i + q;});
}
\end{lstlisting}

The \cpp{mbind} members calls the provided function if expected contains a value, otherwise it forwards the error to the callee. 

Alternative using an existing functor.

\begin{lstlisting}
expected<exception_ptr,int> f1(int i, int j, int k)
{
  return safe_divide(j, k).mbind(bind(add, i, _1)));
}
\end{lstlisting}
\end{frame}
%%%%%%%%%%%%%%%%%%%%%%%%%%%%
\begin{frame}[fragile]
\frametitle{safe\_divide  - exception-based Multiple error conditions}

\begin{lstlisting}
struct NotDivisible: public std::exception {
  int i, j;
  NotDivisible(int i, int j) : i(i), j(j) {}
};

int safe_divide(int i, int j) {
  if (j == 0) throw DivideByZero(); 
  if (i%j != 0) throw NotDivisible(i,j);
  else return i / j; 
}
\end{lstlisting}

\begin{lstlisting}
T divide(T i, T j) {
  try  {    return safe_divide(i,j)  }
  catch(NotDivisible& ex)  {    return ex.i/ex.j;  }
  catch(...)  {    throw;  }
}
\end{lstlisting}
\end{frame}
\subsection{\cpp{divide}}
%%%%%%%%%%%%%%%%%%%%%%%%%%%%
\begin{frame}[fragile]
\frametitle{safe\_divide  - expected-based Multiple error conditions}

\begin{lstlisting}
expected<exception_ptr,int> safe_divide(int i, int j) {
  if (j == 0) return make_unexpected(DivideByZero()); 
  if (i%j != 0) return make_unexpected(NotDivisible(i,j));
  else return i / j; 
}
\end{lstlisting}

\begin{lstlisting}
expected<exception_ptr,int> divide(int i, int j) {
  auto e = safe_divide(i,j);
  if(e.has_exception<NotDivisible>()) return i/j;
  else return e;
}
\end{lstlisting}
\end{frame}
%%%%%%%%%%%%%%%%%%%%%%%%%%%%
\begin{frame}[fragile]
\frametitle{safe\_divide  - expected-based Multiple error conditions}


\begin{lstlisting}
expected<exception_ptr,int> divide(int i, int j) {
  return safe_divide(i,j)
  .catch_exception<NotDivisible>([](auto &ex)
    return ex.i/ex.j;
  });
}
\end{lstlisting}

\end{frame}
\subsection{\cpp{i/k + j/k}}
%%%%%%%%%%%%%%%%%%%%%%%%%%%%
\begin{frame}[fragile]
\frametitle{\cpp{i/k + j/k}  - scale}

\begin{itemize}
  \item exception-based.
\end{itemize}

\begin{lstlisting}
int f2(int i, int j, int k) {
  return safe_divide(i,k) + safe_divide(j,k);
}
\end{lstlisting}

\begin{itemize}
  \item expected-based.
\end{itemize}

\begin{lstlisting}
expected<exception_ptr,int> f2(int i, int j, int k) {
  auto q1 = safe_divide(i, k);
  if (!q1) return make_unexpected(q1);

  auto q2 = safe_divide(j, k);
  if (!q2) return make_unexpected(q2);

  return *q1 + *q2;
}
\end{lstlisting}

\end{frame}
%%%%%%%%%%%%%%%%%%%%%%%%%%%%
\begin{frame}[fragile]
\frametitle{\cpp{i/k + j/k}  - scale}

\begin{itemize}
  \item expected-based functional style.
\end{itemize}

\begin{lstlisting}
expected<exception_ptr,int> f(int i, int j, int k) {
  return safe_divide(i, k).mbind([=](int q1) {
      return safe_divide(j,k).mbind([=](int q2) {
        return q1+q2;
      });
    });
}
\end{lstlisting}
\begin{lstlisting}
auto add = [](int s1, int s2){ return s1+s2; };
expected<exception_ptr,int> f(int i, int j, int k){
  return fmap(add, safe_divide(i, k), safe_divide(j, k));
}
\end{lstlisting}

\end{frame}
\subsection{do-expression}
%%%%%%%%%%%%%%%%%%%%%%%%%%%%
\begin{frame}[fragile]
\frametitle{language-like style  - error propagation}
C++ language extension: Based on the Haskell do-expression. Something similar to the \cpp{await} expression for futures.  

\begin{lstlisting}
do_expression ::= do_initialization ':' do_expression_or_expression
do_expression_or_expression := 
    do_expression | expression
do_initialization := type var '<-' expression
\end{lstlisting}

\begin{itemize}
  \item Transformation semantics.
\end{itemize}

\begin{lstlisting}
[[do_expression]] =
  expression.mbind([&](type var) {
    return [[do_expression_or_expression]]   });
\end{lstlisting}

\end{frame}
%%%%%%%%%%%%%%%%%%%%%%%%%%%%
\begin{frame}[fragile]
\frametitle{\cpp{i/k + j/k}  - do-expression}

\begin{lstlisting}
expected<exception_ptr,int> f2(int i, int j, int k) {
  return ( auto s1 <- safe_divide(i, k) :
           auto s2 <- safe_divide(j, k) :
           s1 + s2  
  );
}
\end{lstlisting}
\begin{itemize}
  \item Transformation semantics.
\end{itemize}
\begin{lstlisting}
expected<exception_ptr,int> f2(int i, int j, int k) {
  return safe_divide(i, k).mbind([&](auto s1) {
    return safe_divide(j, k).mbind([&](auto s2) {
      return s1 + s2;
    });
  }); 
}
\end{lstlisting}
\end{frame}

%%%%%%%%%%%%%%%%%%%%%%%%%%%%
\begin{frame}[fragile]
\frametitle{\cpp{i/k + j/k}  - DO-macro}

\begin{lstlisting}
expected<exception_ptr,int> f2(int i, int j, int k) {
  return DO (
    ( s1, safe_divide(i, k) )
    ( s2, safe_divide(j, k) )
    s1 + s2 
  );
}
\end{lstlisting}

\end{frame}


%%%%%%%%%%%%%%%%%%%%%%%%%%%%
\begin{frame}[fragile]
\frametitle{\cpp{i/k + j/k}  - EXPECT-macro}

\begin{lstlisting}
#define EXPECT(V, EXPR) \
auto BOOST_JOIN(expected,V) = EXPR; \
if (!  BOOST_JOIN(expected,V)) 
  return make_unexpected(BOOST_JOIN(expected,V).error()); \
auto V =*BOOST_JOIN(expected,V)
\end{lstlisting}

\begin{lstlisting}
expected<exception_ptr,int> f2(int i, int j, int k) {
  EXPECT(s1, safe_divide(i, k) );
  EXPECT(s2, safe_divide(j, k));
  return s1 + s2;
}
\end{lstlisting}

\end{frame}


%%%%%%%%%%%%%%%%%%%%%%%%%%%%
\subsection{getIntRange }
%%%%%%%%%%%%

\begin{frame}[fragile]
\frametitle{getIntRange - exception based}

Given
\begin{lstlisting}
  int getInt(istream_range& r);
  void matchesString(std::string, istream_range& r);
\end{lstlisting}

\begin{lstlisting}
pair<int,int> getIntRange(istream_range& r) 
{
    auto f = getInt(r);
             matchesString("..", r);
    auto l = getInt(r);       
    return make_pair(f, l);
}
\end{lstlisting}

\end{frame}
%%%%%%%%%%%%%%%%%%%%%%%%%%%%
\begin{frame}[fragile]
\frametitle{getIntRange - expect based}

Given
\begin{lstlisting}
  expected<errc, int> getInt(istream_range& r);
  expected<errc, void> matchesString(istream_range& r);
\end{lstlisting}

\begin{lstlisting}
expected<errc, pair<int, int>> getIntRange(istream_range& r) 
{
  return 
    auto  f <- getInt(r) : 
               matchedString("..", r) : 
    auto  l <- getInt(r) :
    std::make_pair(f, l);
}
\end{lstlisting}

\end{frame}


%%%%%%%%%%%%%%%%%%%%%%%%%%%%

\section{Behind the scenes}
%%%%%%%%%%%%

\subsection{Unexpected Type}
%%%%%%%%%%%%

%%%%%%%%%%%%%%%%%%%%%%%%%%%%
\begin{frame}[fragile]
\frametitle{class unexpected\_type<E>}

Between explicit and implicit conversion.

Building an explicit type that is implicitly convertible to another type.

E.g. \cpp{imaginary<T>} is explicitly convertible from a \cpp{T} and \cpp{complex<T>} could be implicitly convertible from \cpp{imaginary<T>}. 

\begin{lstlisting}
template <class E=std::exception_ptr>
class unexpected_type {
public:
    unexpected_type() = delete;
    constexpr explicit unexpected_type(E e) 
        : error_(e) {}   
    constexpr E value() const  { return error_; }                              
}; 
\end{lstlisting}

\end{frame}
%%%%%%%%%%%%%%%%%%%%%%%%%%%%
\begin{frame}[fragile]
\frametitle{class unexpected\_type<std::exception\_ptr> specialization}

\begin{lstlisting}
template <>
class unexpected_type<std::exception_ptr> {
    E error_;
public:
    unexpected_type() = delete;
    explicit unexpected_type(std::exception_ptr e) 
        :  error_(e) {}
    template <class E>
    explicit unexpected_type(E e) 
         : error_(std::make_exception_ptr(e)) {}     
    std::exception_ptr value() const { return error_; }                              
}; 
\end{lstlisting}

\end{frame}
%%%%%%%%%%%%%%%%%%%%%%%%%%%%
\begin{frame}[fragile]
\frametitle{unexpected factories}

\begin{lstlisting}
template <class E>
constexpr unexpected_type<decay_t<E>> make_unexpected(E v)  {
    return unexpected_type<decay_t<E>> (ex);
}

unexpected_type<> 
  make_unexpected_from_current_exception() {
    return unexpected_type<> (std::current_exception());
}

template <class T, class E>
constexpr 
unexpected_type<E> make_unexpected(expected<E, T> v)  {
    return unexpected_type<E>(error());
}

\end{lstlisting}
\end{frame}

\subsection{Expected Type}
%%%%%%%%%%%%

%%%%%%%%%%%%%%%%%%%%%%%%%%%%
\begin{frame}[fragile]
\frametitle{expected<E, T> conversions from/to unexpected\_type<E>}


\begin{lstlisting}
template <class E, class T>
class expected {
  // ...
  constexpr expected(unexpected_type<E>&& e);
  expected&  operator=(unexpected_type<E>&& e);
  
};
\end{lstlisting}

\end{frame}

%%%%%%%%%%%%%%%%%%%%%%%%%%%%
\begin{frame}[fragile]
\frametitle{expected<E, T> conversions from/to unexpected\_type<E>}

\begin{lstlisting}
template <class T>
class expected<std::exception_ptr,T> {
  // ...
  constexpr expected(std::exception_ptr&& e);
  expected&  operator=(std::exception_ptr&& e);
  template <class E>
  constexpr expected(unexpected_type<E>&& e);
  template <class E>
  expected&  operator=(unexpected_type<E>&& e);

  // ...  
  constexpr unexpected_type<std::exception_ptr> 
};
\end{lstlisting}

\end{frame}
%%%%%%%%%%%%%%%%%%%%%%%%%%%%
\begin{frame}[fragile]
\frametitle{expected factories}

\begin{lstlisting}
template<typename T>
constexpr expected<std::exception_ptr, decay_t<T> > 
  make_expected(T&& v ) ;
expected<std::exception_ptr, void> make_expected();
template <typename T, typename E>
expected<std::exception_ptr, T> 
  make_expected_from_exception(E e)
template <typename T, typename E>
constexpr expected<decay_t<E>, T> 
  make_expected_from_error(E e);
\end{lstlisting}
\begin{lstlisting}
auto e1 = make_expected(2); // expected<exception_ptr,int>
auto e2 = make_expected_from_exception<int>(bad_alloc()); 
		// expected<exception_ptr,int>
auto e3 = make_expected_from_error<int>(errc::invalid); 
		// expected<errc,int>
\end{lstlisting}
\end{frame}
%%%%%%%%%%%%%%%%%%%%%%%%%%%%
\begin{frame}[fragile]
\frametitle{expected<E> as a type constructor}

\begin{lstlisting}
auto e1 = make_expected<errc>(2); // compile fails
auto e1 = expected<errc,int>(2);  // int is redondant
auto e2 = expected<errc>::make(2);  // no redondant
\end{lstlisting}

\begin{lstlisting}
template <typename E=std::exception_ptr, class T=holder>
class expected;
template <typename E>
class expected<E,holder> {
public:
  template <class T> using type = expected<E,T>;
  template <class T> expected<E,T> make(T&& v) {
    return expected<E,T>(std::forward(v));
  }
};
\end{lstlisting}
\end{frame}
%%%%%%%%%%%%%%%%%%%%%%%%%%%%
\begin{frame}[fragile]
\frametitle{Heterogeneous type factories}

\begin{lstlisting}
auto e1 = make_expected(2); 
auto e2 = expected<>::make(2); 
auto e2 = expected<errc>::make(2);

auto o = make_optional(2); 
auto f = make_ready_future(2); 
\end{lstlisting}

\begin{itemize}
  \item Non-uniform syntax -> no generic algorithms 
\end{itemize}

\end{frame}
%%%%%%%%%%%%%%%%%%%%%%%%%%%%
\begin{frame}[fragile]
\frametitle{Uniform type factories - type constructor}

\begin{itemize}
  \item Type constructor 
\end{itemize}

\begin{lstlisting}
auto e2 = make<expected<>>(2); 
auto e2 = make<expected<errc>>(2);
auto o = make<optional<>>(2); 
auto f = make<future<>>(2); 
\end{lstlisting}
\end{frame}

%%%%%%%%%%%%%%%%%%%%%%%%%%%%
\begin{frame}[fragile]
\frametitle{\cpp{expected<E>} as a type constructor}

\begin{lstlisting}
auto e = make<expected<errc>>(2); 
// e has type expected<errc, int>
auto e = make<expected<>>(2); 
// e has type expected<exception_ptr, int>
\end{lstlisting}

\begin{lstlisting}
template<class F, class... Args>
using apply = typename F::template type<Args...>;
\end{lstlisting}
  
\begin{lstlisting}
template<typename TC, class T>
constexpr apply<TC,decay_t<T>> make(T&& v )  {
  return apply<TC,decay_t<T>>(std::forward(v))
}
\end{lstlisting}

\end{frame}
%%%%%%%%%%%%%%%%%%%%%%%%%%%%
\begin{frame}[fragile]
\frametitle{Uniform type factories - Lifting }

\begin{itemize}
  \item Lifting variadic template class to create a type constructor 
\end{itemize}

\begin{lstlisting}
template <template <class ...> class F, class... Args>
struct lift {
  template<class... Args2> using type = F<Args..., Args2...>;
};
\end{lstlisting} 

\begin{lstlisting}
auto e2 = make<lift<expected, exception_ptr>>(2); 
auto e2 = make<lift<expected, errc>>(2);
auto o = make<lift<optional>>(2); 
auto f = make<lift<future>>(2); 
\end{lstlisting}

\end{frame}
%%%%%%%%%%%%%%%%%%%%%%%%%%%%
\begin{frame}[fragile]
\frametitle{\cpp{lift<future>} as a  type constructor}

\begin{itemize}
  \item However, \cpp{future<T>} is not explicitly constructible from \cpp{T}.
\end{itemize}

\begin{lstlisting}
auto f = make_ready_future(2); 
auto f = make<lift<future>>(2); 
\end{lstlisting}

\begin{itemize}
  \item We need an indirection.
\end{itemize}

\begin{lstlisting}
template<class M, class T, 
         class Mo = apply<M, decay_t<T>> 
         >
Mo make(T&& v )  {
    return Mo::make(std::forward<T>(v));
}
\end{lstlisting}

\end{frame}
%%%%%%%%%%%%%%%%%%%%%%%%%%%%
\begin{frame}[fragile]
\frametitle{\cpp{future<T>} mapping}

\begin{lstlisting}
template<class U>
struct monad_traits<future<U>> { 
  template<class M, class T> 
  future<decay_t<T>> make(T&& v )  {
    return make_ready_future(std::forward<T>(v));
  }
  // ...
};
\end{lstlisting}

\end{frame}

%%%%%%%%%%%%%%%%%%%%%%%%%%%%
\begin{frame}[fragile]
\frametitle{Differences between \cpp{expected<errc, T>} and \cpp{expected<exception_ptr, T>}}

\begin{tabular}{|c|c|c|}
\hline
                    & \cpp{expected<errc, T>} & \cpp{expected<exception_ptr, T>}  \\
\hline
\textbf{never-empty warranty} & \cpp{E()} & almost yes \\
\hline
\textbf{has\_exception} & no & yes \\
\hline
\textbf{catch\_exception} & no & yes \\
\hline
\textbf{all non valued equal} & no & yes \\
\hline
\end{tabular}
\end{frame}

\subsection{Related types}
%%%%%%%%%%%%

%%%%%%%%%%%%%%%%%%%%%%%%%%%%
\begin{frame}[fragile]
\frametitle{\cpp{boost::variant<unexpected_type<E>,T>} Comparison}

\begin{tabular}{|c|c|c|}
\hline
                    & \textbf{variant} & \textbf{expected}  \\
\hline
\textbf{never-empty warranty} & yes & almost yes \\
\hline
\textbf{factories} & no & yes  \\
\hline
\textbf{value\_type} & no & yes  \\
\hline
\textbf{default constructor} & yes (if E is ) & yes (if E is )  \\
\hline
\textbf{observers} & get<T>/get<E> & value / error   \\
\hline
\textbf{visitation} & apply\_visitor & fmap/mbind/catch\_error  \\
\hline
\end{tabular}
\end{frame}

%%%%%%%%%%%%%%%%%%%%%%%%%%%%
\begin{frame}[fragile]
\frametitle{\cpp{std::future<T>} Comparison}

\begin{tabular}{|c|c|c|}
\hline
                    & \textbf{future} & \textbf{expected}  \\
\hline
\textbf{specific null value} & no & no \\
\hline
\textbf{relational operators} & no & yes \\
\hline
\textbf{factories} & make\_ready\_future & make\_expected  \\
\hline
\textbf{emplace} & no & yes \\
\hline
\textbf{value\_type} & no & yes  \\
\hline
\textbf{default constructor} & yes(invalid) & yes (E() )  \\
\hline
\textbf{observers} & get / wait & operator* / value / error   \\
\hline
\textbf{visitation} & then & fmap/mbind/catch\_error  \\
\hline
\textbf{grouping}  &  when\_all / when\_any &  n.a. \\
\hline
\end{tabular}
\end{frame}

%%%%%%%%%%%%%%%%%%%%%%%%%%%%
\begin{frame}[fragile]
\frametitle{Conversions from/to ready std::experimental::future<T> }

\begin{lstlisting}
template <class T>
expected<exception_ptr,T> make_expected(future<T>&& f) {
  assert (f.ready() && "future not ready");
  try {
    return f.get();
  } catch (...) {
    return make_unexpected_from_exception();
  }
}
\end{lstlisting}

\begin{lstlisting}
template <class T>
std::future<T> make_ready_future(expected<exception_ptr, T>&& e) {
  if (e) return make_ready_future(*e);
  else return make_unexpected_future<T>(e.error()); 
}
\end{lstlisting}
\end{frame}

%%%%%%%%%%%%%%%%%%%%%%%%%%%%
\begin{frame}[fragile]
\frametitle{make\_unexpected\_future }

\begin{lstlisting}
template <class T, class E>
std::future<T> make_unexpected_future(E e)  {
  std::promise<T> p;
  std::future<T> f = p.get_future();
  p.set_exception(std::make_exception_ptr(e));
  return std::move(f);
}
\end{lstlisting}

\end{frame}

%%%%%%%%%%%%%%%%%%%%%%%%%%%%
\begin{frame}[fragile]
\frametitle{make\_expected alternative implementation }

Implementations of  \cpp{future<T>} that store the exception on a  \cpp{exception_ptr}, could define this function as a friend function and make it more efficient.   

\begin{lstlisting}
template <class T>
expected<exception_ptr, T> make_expected(future<T>&& f) {
  assert (f.ready() && "future not ready");
  lock_guard<mutex> lk(f.get_mutex());
  if (f.has_value(lk)) return f.get(lk);
  else return make_unexpected(f.get_exception_ptr(lk));
}
\end{lstlisting}

\end{frame}


%%%%%%%%%%%%%%%%%%%%%%%%%%%%
\begin{frame}[fragile]
\frametitle{std::experimental::optional<T> Comparison}

\cpp{expected<E, T>} is as an \cpp{std::experimental::optional} that collapse all the values of \cpp{E} to \cpp{nullopt}.

\begin{tabular}{|c|c|c|}
\hline
                    & \textbf{optional} & \textbf{expected}  \\
\hline
\textbf{specific null value} & yes & no \\
\hline
\textbf{relational operators} & yes & yes \\
\hline
\textbf{swap} & yes & yes \\
\hline
\textbf{factories} & make\_optional & make\_expected  \\
\hline
\textbf{value\_type} & no & yes  \\
\hline
\textbf{default constructor} & yes(nullopt)  & yes (E() )  \\
\hline
\textbf{observers} & value  & value / error   \\
\hline
\textbf{visitation} & no & fmap/mbind/catch\_error  \\
\hline
\end{tabular}
\end{frame}

%%%%%%%%%%%%%%%%%%%%%%%%%%%%
\begin{frame}[fragile]
\frametitle{std::experimental::optional<T> Conversions}
\begin{lstlisting}
template <class T>
optional<T> make_optional(expected<E, T> v) {
  if (v) return make_optional(*v);
  else nullopt;
}
\end{lstlisting}

\begin{lstlisting}
struct conversion_from_nullopt {};
template <class T>
expected<exception_ptr, T> make_expected(optional<T> v) {
  if (v) return make_expected(*v);
  else make_unexpected(conversion_from_nullopt());
}
\end{lstlisting}
\end{frame}

\subsection{Operators}

\section{Functional}
%%%%%%%%%%%%%%%%%%%%%%%%%%%%
\begin{frame}[fragile]
\frametitle{Motivation}

Problem:

\begin{itemize}
  \item We want too define some algorithms that work for \cpp{expected<E,T>}, \cpp{optional<T>} and \cpp{future<T>} between other.
  \item While these share a common underlying semantic, the interface is different.
\end{itemize}

Two alternatives:

\begin{itemize}
  \item Direct: the model must provide the specific operations to conform to the Concept, E.g. the expression \cpp{*e} is well formed.
  \item Indirect: the developer of the model adapts the model to the concept. E.g. the Concept requires that expression \cpp{deref(e)} is well formed and the Model mapping states how this operation is implemented using the Model.
\end{itemize}
\end{frame}
%%%%%%%%%%%%%%%%%%%%%%%%%%%%
\begin{frame}[fragile]
\frametitle{Advantages/Liabilities}

\begin{itemize}
  \item The direct alternative would need to change the interface of the Model to conform to the Concept or,
    \item define the concept in terms of non-member functions.
   \item The indirect alternative is more open but introduces a lot of non-member members. 
   \item To avoid the proliferation of possible clashes, the concept members are defined on a specific namespace associated for the concept. 
\end{itemize}

\end{frame}
%%%%%%%%%%%%%%%%%%%%%%%%%%%%
\begin{frame}[fragile]
\frametitle{Automatic concept mapping?}

Should the mapping be automatic, that is, have a default mapping that is applied implicitly if the type conform to some syntactical requirements? 

\begin{itemize}
  \item While concepts are more than syntax the explicit mapping would be cumbersome. 
\end{itemize}

Could several types share the same mapping?

\begin{itemize}
  \item Several types could already be a model of another Concept that can be used to implement the mapping. 
\end{itemize}


\end{frame}
\subsection{Rebindable}
%%%%%%%%%%%%

\begin{frame}[fragile]
\frametitle{Rebindable features}

Rebindable is a basic concept that provides a way to

\begin{itemize}
  \item \cpp{value_type<M>} : obtain the type of the underlying value, 
  \item \cpp{type_constructor<M>} : obtain the type constructor. 
  \item \cpp{rebind<M,T>} : build another type substituting the underlying value type. 
\end{itemize}

satisfying 

\begin{lstlisting}
is_same<apply<type_constructor<M>, value_type<M>>, M>::value
is_same<type_constructor<apply<TC, U>>, TC>::value
is_same<rebind<M, value_type<M>>, M>::value
is_same<rebind<M, U>, apply<type_constructor<M>,U>>::value
\end{lstlisting}

\end{frame}
%%%%%%%%%%%%%%%%%%%%%%%%%%%%%%%%%%%%
\begin{frame}[fragile]
\frametitle{Rebindable features}

Where

\begin{lstlisting}
template<class F, class... Args>
using apply = typename F::template type<Args...>;
\end{lstlisting}

Examples:
\begin{itemize}
  \item pointers:  \cpp{T*}, 
  \item smart pointers: \cpp{shared_ptr<T>}, 
  \item containers, \cpp{vector<T>},  \cpp{array<T,N>},  \cpp{T[N]} 
  \item wrappers: \cpp{optional<T>}, \cpp{expected<E, T>}, \cpp{future<T>},
  \item other: \cpp{allocator<T>}   
\end{itemize}

\end{frame}

%%%%%%%%%%%%%%%%%%%%%%%%%%%%%%%%%%%%
\subsection{PossiblyValued}
%%%%%%%%%%%%%%%%%%%%%%%%%%%%%%%%%%%%
\begin{frame}[fragile]
\frametitle{PossiblyValued features}

PossibleValued is a refinement of a Rebindable  concept that allows to

\begin{itemize}
  \item \cpp{has_value(m)}: check if a value is present.
  \item \cpp{deref(m)}: obtain a reference to the value. (defined only if \cpp{has_value}).
  \item \cpp{value(m)}: get a reference to the stored value or throw an exception.
  
  \item \cpp{error_type<M>}: obtain the associated error type.
  \item \cpp{error(m)}: get a reference to the stored error (defined only if \cpp{has_value} is false).
  
  \item \cpp{errored_type<M>}: obtain the associated errored type, which is implicitly convertible to the model.
  \item \cpp{make_errored(m)}: get a value storing the stored error that is convertible to the model. 
 
\end{itemize}
\end{frame}
%%%%%%%%%%%%%%%%%%%%%%%%%%%%%%%%%%%%
\begin{frame}[fragile]
\frametitle{PossiblyValued algorithms}

\begin{lstlisting}
template <class PV> 
// requires PossibleValued<PV> 
// && EqualityComparable<value_type<PV>>
constexpr bool equal( const PV& x, const PV& y ) {
  using namespace possibly_valued;
  return has_value(x) != has_value(y) ? false 
       : ( has_value(x) ? deref(x) == deref(y) : true );
}
\end{lstlisting}

\begin{lstlisting}
template <class PV, class U> 
// requires PossibleValued<PV> 
// && Convertible<value_type<PV>,U>
constexpr value_type<PV> value_or( const PV& x, U&& y ) {
  using namespace possibly_valued;
  return has_value(x) ? deref(x) : value_type<PV>(y); 
}
\end{lstlisting}

\end{frame}
%%%%%%%%%%%%%%%%%%%%%%%%%%%%%%%%%%%%
\subsection{Functor}
%%%%%%%%%%%%%%%%%%%%%%%%%%%%%%%%%%%%
\begin{frame}[fragile]
\frametitle{C++ 'Functor'}

\cpp{Functor} is a refinement of Rebindable for type wrappers that can be mapped over.  It provides a

\begin{itemize}
  \item \cpp{fmap(f,m)}: traverse a functor applying a function".
\end{itemize}
        
satisfying

\begin{lstlisting}
bind(fmap,id,_1)  ==  bind(id,_1)
bind(fmap, compose(f, g), _1)  ==  
        compose(bind(fmap, f,_1), bind(fmap, g, _1)
\end{lstlisting}

\end{frame}
%%%%%%%%%%%%%%%%%%%%%%%%%%%%%%%%%%%%
\begin{frame}[fragile]
\frametitle{C++ Functor - Specialization for PossibleValued}
       
Specialization for models of \cpp{PossibleValued}       

\begin{lstlisting}
template <class F, class M>
// requires Functor<M> && Invokable<F, value_type<M>>
rebind<M, result_of_t<F(value_type<M>)> fmap(F&& f, M&& m) {
  using namespace possibly_valued;
  return has_value( forward<M>(m))
  ? forward<F>(f)( deref( forward<M>(m) ) )
  : make_errored(m);
}
\end{lstlisting}
         
\end{frame}

\subsection{Monad}
%%%%%%%%%%%%
\begin{frame}[fragile]
\frametitle{Monad}

\cpp{Monad} is a refinement of \cpp{Functor} concept  providing a way to:

\begin{itemize}
  \item \cpp{make<M>(v)}: build a monad instance from the underlying value type.
  \item \cpp{mbind(m, f)} : binds a function that will be called only if the action associated to the Monad succeeds.
  \item \cpp{mdo(m1, m2)} : sequentially compose two actions, discarding any value produced by the first.
\end{itemize}

satisfying

\begin{lstlisting}
mbind(make<M>(x), f)  ==  f(x)
mbind(m, [](auto x) { return make<M>(x);} )  ==  m
mbind(m, [](auto x) { return mbind(f(x), g); }  ==  
           mbind(mbind(m, f), g)

fmap(f, m)  ==  
  mbind(m, compose([](auto x) { return make<M>(x);}, f))
\end{lstlisting}
\end{frame}

\subsection{MonadError}
%%%%%%%%%%%%%%%%%%%%%%%%%%%%%%%%%%%%
\begin{frame}[fragile]
\frametitle{MonadError}

A MonadError is a Monad which can store errors of different types.

\begin{itemize}
  \item \cpp{make_error<M>(e)}: signal an error on a monadic function.
  \item \cpp{catch_error(m, f)}: possibly recover from previous errors and return to normal execution.
\end{itemize}

Should satisfy the laws:

\begin{lstlisting}
  fmap(make_error<M>(e),  f) == make_error<M>(e)
  mbind(make_error<M>(e),  f) == make_error<M>(e)
  mdo(make_error<M>(e),  f) == make_error<M>(e)
  catch_error(make<M>(v),  f) == make<M>(v)
\end{lstlisting}
\end{frame}

\subsection{MonadException}
%%%%%%%%%%%%%%%%%%%%%%%%%%%%%%%%%%%%
\begin{frame}[fragile]
\frametitle{MonadException}

A MonadException is a MonadError which can store errors of different types.

\begin{itemize}
  \item \cpp{make_exception(e)}: signal an exception on a monadic function.
  \item \cpp{has_exception<E>()} : check if the stored exception is of the type parameter \cpp{E}.
  \item \cpp{catch_exception<E>(m, f)}: possibly recover from previous exception and return to normal execution.
\end{itemize}

Examples
\begin{itemize}
  \item \cpp{expected<exception_ptr,T>} 
  \item \cpp{expected<any,T>} 
  \item \cpp{expected<variant<E1, ..., En>,T>} 
  \item \cpp{future<T>} 
  \item \cpp{shared_future<T>} 
\end{itemize}
\end{frame}
%%%%%%%%%%%%%%%%%%%%%%%%%%%%%%%%%%%%
\begin{frame}[fragile]
\frametitle{MonadException catch\_exception example}

\begin{lstlisting}
template <class M> apply<M, int> safe_divide(int i, int j)  
{
  using namespace boost::functional::monad_exception;
  if (j == 0) return make_exception<M>(DivideByZero());
  else return make<M>(i / j);
}
template <class M> apply<M, int> divide(int i, int j)  
{
  using namespace boost::functional::monad_exception;
  return catch_exception<NotDivisible>(
    safe_divide<M>(i,j), 
    [](auto& e)  { return make<M>(e.i / e.j);  }
  );
}
//...
  auto a = divide<expected<>>(1, 0);
\end{lstlisting}
\end{frame}
%%%%%%%%%%%%%%%%%%%%%%%%%%%%%%%%%%%%

\section{Open Points, Future Work and Conclusions}
%%%%%%%%%%%

\begin{frame}
\frametitle{Open Points ...}
%%%%%%%%%%%%

\begin{itemize}
  \item Should the \cpp{operator==} collapse all the errors?
  \item Should \cpp{expected<errc,T>} and \cpp{expected<exception_ptr,T>} be represented by two different classes?
  \item Should \cpp{expected<E,T>} throw E?
\end{itemize}

\end{frame}

\begin{frame}
\frametitle{Future Work ...}
%%%%%%%%%%%%

\begin{itemize}
  \item Allocator support for expected.
  \item Define a common visitation interface for \cpp{any}, \cpp{variant<E1, ..., En>}, \cpp{exception_ptr}.
  \item Work on a separated Monads library.
\end{itemize}

\end{frame}

\begin{frame}
\frametitle{Conclusions}
%%%%%%%%%%%%
\begin{itemize}
  \item \cpp{expected<E, T>} monadic functions are useful tools, 
  \item that allows to combine functions that return \cpp{expected<E, T>} but,
  \item it would be much easier to use it with a specific do-expression.
\end{itemize}

\begin{itemize}
  \item \cpp{expected<E, T>}, \cpp{future<T>} and \cpp{optional<T>} share a lot of things but have some differences.
  \item Defining a common interface for the functions that have the same behavior allows us to define generic algorithms on top of these concepts.
  \item Having a monadic common interface would be one step towards this goal.
\end{itemize}

\end{frame}

\begin{frame}
%%%%%%%%%%%%
\begin{center}
\Large{Thanks for your attention!}
\end{center}
\begin{center}
\Large{Questions?}
\end{center}
\end{frame}



\end{document}
