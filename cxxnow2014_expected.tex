\documentclass[xcolor=dvipsnames]{beamer}

\usepackage[utf8]{inputenc}
\usepackage[english]{babel}
\usepackage{url}
\usepackage{lmodern}
\usepackage{listings}
\usepackage{graphicx}
\usepackage{xcolor}
\usepackage{textcomp} 
\usepackage{hyperref}
\usepackage{subcaption}
\usepackage[T1]{fontenc}

\usepackage{color}
 
\definecolor{dkgreen}{rgb}{0,0.6,0}
\definecolor{gray}{rgb}{0.5,0.5,0.5}
\definecolor{mauve}{rgb}{0.58,0,0.82}
 
\lstset{
  language=c++,                % the language of the code
  basicstyle=\footnotesize,       % the size of the fonts that are used for the code
  numbers=left,                   % where to put the line-numbers
  numberstyle=\tiny\color{gray},  % the style that is used for the line-numbers
  stepnumber=1,                   % the step between two line-numbers. If it's 1, each line 
                                  % will be numbered
  numbersep=5pt,                  % how far the line-numbers are from the code
  backgroundcolor=\color{white},      % choose the background color. You must add \usepackage{color}
  showtabs=false,                 % show tabs within strings adding particular underscores
  frame=single,                   % adds a frame around the code
  rulecolor=\color{black},        % if not set, the frame-color may be changed on line-breaks within not-black text (e.g. comments (green here))
  tabsize=2,                      % sets default tabsize to 2 spaces
  captionpos=b,                   % sets the caption-position to bottom
  breaklines=true,                % sets automatic line breaking
  breakatwhitespace=false,        % sets if automatic breaks should only happen at whitespace
  title=\lstname,                   % show the filename of files included with \lstinputlisting;
                                  % also try caption instead of title
  keywordstyle=\color{blue},          % keyword style
  commentstyle=\color{dkgreen},       % comment style
  stringstyle=\color{mauve},         % string literal style
  morekeywords={expect, await, constexpr, explicit}
}

\newcommand{\cpp}[1]{\lstinline{#1}}

\setbeamertemplate{footline}{\hspace*{.5cm}\scriptsize{\insertauthor\hspace*{50pt} \hfill\insertframenumber\hspace*{.5cm}}}

\setbeamertemplate{section in toc}{%
\textcolor{MidnightBlue}{$\blacktriangleright$ \inserttocsection}
}

\usecolortheme{seahorse}
\usecolortheme{rose}
\useoutertheme{infolines}

\usecolortheme[named=SkyBlue]{structure}
\setbeamercolor{block title}{fg=MidnightBlue}

\AtBeginSection[]
{
  \begin{frame}<beamer>{Content}
    \tableofcontents[currentsection,hideothersubsections]
  \end{frame}
} 

\title{\textsc{Expected}: Exception-friendly Error Monad}
\subtitle{C++Now 2014}
\author[\textsc{Vicente Botet}]{Vicente \textsc{Botet Escriba} \texttt{vicente.botet@wanadoo.fr}}
\institute[Alcatel-Lucent]{Alcatel-Lucent International-Lannion}
\date[]{2014, May 16}

\begin{document}
\maketitle

\section{Introduction}
%%%%%%%%%%

\subsection{About me}

\begin{frame}
\frametitle{About Vicente Botet Escriba}

\begin{figure}[p]
  \centering
  \begin{subfigure}[b]{0.3\textwidth}
    \includegraphics[scale=0.3]{images/vicente_botet.png}
  \end{subfigure}
  \qquad \qquad \quad
  \begin{subfigure}[b]{0.3\textwidth}
    \includegraphics[scale=0.3]{images/alu.jpg}
  \end{subfigure}
\end{figure}

\begin{itemize}
\item Spanish.
\item Work for Alcatel-Lucent International in Lannion - France.
\item Interest in software engineering, language design, concurrency and STM.
\end{itemize}

\begin{block}{Open-source involvement}
\begin{itemize}
\item \textbf{Boost C++ library} Co-author and maintainer of Boost.Ratio, Boost.Chrono, Boost.Thread and some utilities in Boost.Utility.
\item \textbf{Toward Boost C++ library} author of a lot of unfinished prototypes.
\end{itemize}
\end{block}
\end{frame}

\subsection{About Boost.Expected}
%%%%%%%%%%%%
\begin{frame}
\frametitle{About Boost.Expected}

\begin{itemize}
\item Original idea of Andrei Alexandrescu (Systematic Error Handling in C++).
\item Interface based on the new \cpp{std::experimental::optional<T>} type.
\item Extended to conform to the error monad concept.
\item We are currently working on a C++ standard proposal that we expect to be ready for Rapperswil.
\item Once the interface would be more stable we will propose it for review to Boost.
\end{itemize}
\end{frame}

\begin{frame}
\frametitle{About Boost.Expected}

\begin{itemize}
\item The project born as a proposal the authors made for GSoC 2013.
\item Pierre Talbot is the student working on it with me.
\end{itemize}

\begin{block}{Links}
\begin{itemize}
\item Git repository : https://github.com/ptal/Boost.Expected
\item C++ draft proposal : http://www.hyc.io/boost/expected-proposal.pdf
\end{itemize}
\end{block}
\end{frame}
%%%%%%%%%%%%%%%%%%%%%%%%%%%%

\begin{frame}[fragile]
\frametitle{Overview}

\cpp{expected<E, T>} answer to the question "do you contain a value of type \cpp{T} or an error of type \cpp{E}?".

\begin{lstlisting}
template <class E, class T>
class expected {
  bool has_value_;
  union{ 
    T val_;
    E err_;
  };
};
\end{lstlisting}

\begin{itemize}
\item \cpp{expected<E,T> = T + unexpected_type<E>}
\item \cpp{T != unexpected_type<E>}
\item We want that \cpp{expected<E,T>} behaves as \cpp{T}.
\item \cpp{expected<std::exception_ptr,T> = Expected<T>}
\end{itemize}

\end{frame}

%%%%%%%%%%%%%%%%%%%%%%%%%%%%
\begin{frame}[fragile]
\frametitle{How to create an expected object}

\begin{itemize}
\item Valued.
\end{itemize}

\begin{lstlisting}
expected<exception_ptr, int> ei = make_expected(1); 
expected<exception_ptr, int> ej{2}; // alternative syntax
expected<exception_ptr, int> ek = ej; // ek has value 2

expected<exception_ptr,MoveOnly> el{in_place, "arg"};        
expected<exception_ptr,MoveOnly> el{expect, "arg"};        
  // calls MoveOnly{"arg"} in place
\end{lstlisting}

\end{frame}
%%%%%%%%%%%%%%%%%%%%%%%%%%%%
\begin{frame}[fragile]
\frametitle{How to create an unexpected object}

\begin{itemize}
\item Unexpected.
\end{itemize}

\begin{lstlisting}
expected<errc, int> ek = make_unexpected(errc::invalid);

expected<string,int> em{unexpect, "arg"};        
  // unexpected value, calls string{"arg"} in place

expected<string,int> e;  // e is unexpected...
ei = {}; // reset to unexpected
\end{lstlisting}

\end{frame}

%%%%%%%%%%%%%%%%%%%%%%%%%%%%
\begin{frame}[fragile]
\frametitle{Observing an expected object}

\begin{itemize}
\item Explicit conversion to bool.
\item \cpp{operator*()} narrow contract
\end{itemize}

\begin{lstlisting}
if (expected<exception_ptr, char> ch = readNextChar()) {
  process_char(*ch);
}
\end{lstlisting}

\begin{itemize}
\item \cpp{value()} wide contract
\end{itemize}

\begin{lstlisting}
expected<errc,int> ei = str2int(s);
try {
    process_int(ei.value());
}  catch(bad_expected_access const&) {
    std::cout << "this was not a number";
}
\end{lstlisting}

\end{frame}

%%%%%%%%%%%%%%%%%%%%%%%%%%%%
\section{Expected Library by Example}
%%%%%%%%%%%%%%%%%%%%%%%%%%%%
\subsection{\cpp{safe_divide}}
%%%%%%%%%%%%%%%%%%%%%%%%%%%%
\begin{frame}[fragile]
\frametitle{safe\_divide exception-based style}

\begin{lstlisting}
struct DivideByZero: public std::exception {...};
\end{lstlisting}

\begin{lstlisting}
int safe_divide(int i, int j)
{
  if (j==0) throw DivideByZero();
  else return i / j;
}
\end{lstlisting}
\end{frame}
%%%%%%%%%%%%%%%%%%%%%%%%%%%%
\begin{frame}[fragile]
\frametitle{safe\_divide Expected-based style}

\begin{lstlisting}
Expected<int> safe_divide(int i, int j)
{
  if (j==0) return Expected<int>::from_error(DivideByZero());
  else return Expected<int>(i / j);
}
\end{lstlisting}
\begin{itemize}
  \item (3,4) redondant use of  \cpp{Expected<int>}. 
\end{itemize}

\end{frame}
%%%%%%%%%%%%%%%%%%%%%%%%%%%%
\begin{frame}[fragile]
\frametitle{safe\_divide expected-based style}

Using \cpp{boost::expected<exception_ptr, int>} to carry the error condition.

\begin{lstlisting}
expected<exception_ptr,int> safe_divide(int i, int j)
{
  if (j==0) return make_unexpected(DivideByZero());
  else return i / j;
}
\end{lstlisting}

\begin{itemize}
  \item (3) uses implicit conversion from \cpp{unexpected_type<E>} to \cpp{expected<E, T>}. 
  \item (4) uses implicit conversion from \cpp{T} to \cpp{expected<E, T>}.
\end{itemize}

\begin{itemize}
\item The advantages are that we have a clean way to fail without using the exception machinery, and we can give precise information about why it failed as well. 
\item The liability is that this function is going to be tedious to use.
\end{itemize}

\end{frame}
%%%%%%%%%%%%%%%%%%%%%%%%%%%%
\begin{frame}[fragile]
\frametitle{safe\_divide expected-style  - getting the value or an exception}

\begin{itemize}
  \item The caller obtains the stored value using the member \cpp{value()} function.
\end{itemize}

\begin{lstlisting}
  auto x = safe_divide(i,j).value(); // throw on error
\end{lstlisting}

\begin{itemize}
  \item The caller can also store the expected value and check if a value is stored using the bool conversion  function.
\end{itemize}

\begin{lstlisting}
  auto x = safe_divide(i,j); // won't throw
  if (! x) 
    int i = x.value(); // just "re"throw the stored exception
\end{lstlisting}

\end{frame}
%%%%%%%%%%%%%%%%%%%%%%%%%%%%
%%%%%%%%%%%%%%%%%%%%%%%%%%%%
\begin{frame}[fragile]
\frametitle{\cpp{i + j/k} expected-style  - getting the value when available}

\begin{itemize}
  \item When the user knows that there is a value it can use the narrowed contract \cpp{operator*()}.
\end{itemize}

\begin{lstlisting}
expected<exception_ptr,int> f1(int i, int j, int k)
{
  auto q = safe_divide(j, k)
  if(q) return i + *q;
  else return make_unexpected(q);
}
\end{lstlisting}

\end{frame}
\subsection{\cpp{i + j/k}}
%%%%%%%%%%%%%%%%%%%%%%%%%%%%
\begin{frame}[fragile]
\frametitle{\cpp{i + j/k}  - functional style}

\begin{lstlisting}
expected<exception_ptr,int> f1(int i, int j, int k)
{
  return safe_divide(j, k).fmap([i](int q){return i + q;});
}
\end{lstlisting}

\cpp{fmap}  calls the provided function if expected contains a value and wraps the result, otherwise it forwards the error to the callee. 

Alternative using an existing functor.

\begin{lstlisting}
expected<exception_ptr,int> f1(int i, int j, int k)
{
  return safe_divide(j, k).fmap(bind(add, i, _1)));
}
\end{lstlisting}
\end{frame}
%%%%%%%%%%%%%%%%%%%%%%%%%%%%
\begin{frame}[fragile]
\frametitle{safe\_divide  - exception-based Multiple error conditions}

\begin{lstlisting}
struct NotDivisible: public std::exception {
  int i, j;
  NotDivisible(int i, int j) : i(i), j(j) {}
};

int safe_divide(int i, int j) {
  if (j == 0) throw DivideByZero(); 
  if (i%j != 0) throw NotDivisible(i,j);
  else return i / j; 
}
\end{lstlisting}

\begin{lstlisting}
T divide(T i, T j) {
  try  {    return safe_divide(i,j)  }
  catch(NotDivisible& ex)  {    return ex.i/ex.j;  }
  catch(...)  {    throw;  }
}
\end{lstlisting}
\end{frame}
\subsection{\cpp{divide}}
%%%%%%%%%%%%%%%%%%%%%%%%%%%%
\begin{frame}[fragile]
\frametitle{safe\_divide  - expected-based Multiple error conditions}

\begin{lstlisting}
expected<exception_ptr,int> safe_divide(int i, int j) {
  if (j == 0) return make_unexpected(DivideByZero()); 
  if (i%j != 0) return make_unexpected(NotDivisible(i,j));
  else return i / j; 
}
\end{lstlisting}

\begin{lstlisting}
expected<exception_ptr,int> divide(int i, int j) {
  auto e = safe_divide(i,j);
  if(e.has_exception<NotDivisible>()) return i/j;
  else return e;
}
\end{lstlisting}
\end{frame}
%%%%%%%%%%%%%%%%%%%%%%%%%%%%
\begin{frame}[fragile]
\frametitle{safe\_divide  - expected-based Multiple error conditions}

\begin{lstlisting}
expected<exception_ptr,int> divide(int i, int j) {
  return safe_divide(i,j)
  .catch_exception<NotDivisible>([](auto &ex)
    return ex.i/ex.j;
  });
}
\end{lstlisting}

\end{frame}
\subsection{\cpp{i/k + j/k}}
%%%%%%%%%%%%%%%%%%%%%%%%%%%%
\begin{frame}[fragile]
\frametitle{\cpp{i/k + j/k}  - scale}

\begin{itemize}
  \item exception-based.
\end{itemize}

\begin{lstlisting}
int f2(int i, int j, int k) {
  return safe_divide(i,k) + safe_divide(j,k);
}
\end{lstlisting}

\begin{itemize}
  \item expected-based.
\end{itemize}

\begin{lstlisting}
expected<exception_ptr,int> f2(int i, int j, int k) {
  auto q1 = safe_divide(i, k);
  if (!q1) return make_unexpected(q1);

  auto q2 = safe_divide(j, k);
  if (!q2) return make_unexpected(q2);

  return *q1 + *q2;
}
\end{lstlisting}

\end{frame}
%%%%%%%%%%%%%%%%%%%%%%%%%%%%
\begin{frame}[fragile]
\frametitle{\cpp{i/k + j/k}  - scale}

\begin{itemize}
  \item expected-based functional style.
\end{itemize}

\begin{lstlisting}
expected<exception_ptr,int> f(int i, int j, int k) {
  return safe_divide(i, k).mbind([=](int q1) {
      return safe_divide(j,k).fmap([=](int q2) {
        return q1+q2;
      });
    });
}
\end{lstlisting}

\cpp{mbind} calls the provided function if expected contains a value, otherwise it forwards the error to the callee. 

\end{frame}
%%%%%%%%%%%%%%%%%%%%%%%%%%%%
\begin{frame}[fragile]
\frametitle{\cpp{i/k + j/k}  - scale}

\begin{lstlisting}
auto add = [](int s1, int s2){ return s1+s2; };
expected<exception_ptr,int> f(int i, int j, int k){
  return fmap(add, safe_divide(i, k), safe_divide(j, k));
}
\end{lstlisting}

The \cpp{fmap} non-member function calls the provided function if all the expected contains a value and wraps the result, otherwise it forwards the first error to the callee. 

\end{frame}
\subsection{do-expression}
%%%%%%%%%%%%%%%%%%%%%%%%%%%%
\begin{frame}[fragile]
\frametitle{language-like style  - error propagation}
Possible C++ language extension: Based on the Haskell do-expression. Something similar to the proposed \cpp{await} expression for futures.  

\begin{lstlisting}
expected<exception_ptr,int> f2(int i, int j, int k) {
  return ( auto s1 <- safe_divide(i, k) :
           auto s2 <- safe_divide(j, k) :
           s1 + s2  
  );
}
\end{lstlisting}

transformed in 

\begin{lstlisting}
expected<exception_ptr,int> f2(int i, int j, int k) {
  return mbind(safe_divide(i, k),[&](auto s1) {
    return mbind(safe_divide(j, k),[&](auto s2) {
      return s1 + s2;
    });
  }); 
}
\end{lstlisting}
\end{frame}
%%%%%%%%%%%%%%%%%%%%%%%%%%%%
\begin{frame}[fragile]
\frametitle{\cpp{i/k + j/k}  - DO-macro}

\begin{lstlisting}
expected<exception_ptr,int> f2(int i, int j, int k) {
  return DO (
    ( s1, safe_divide(i, k) )
    ( s2, safe_divide(j, k) )
    s1 + s2 
  );
}
\end{lstlisting}

\end{frame}
%%%%%%%%%%%%%%%%%%%%%%%%%%%%
\begin{frame}[fragile]
\frametitle{\cpp{i/k + j/k}  - EXPECT-macro}

\begin{lstlisting}
#define EXPECT(V, EXPR) \
auto BOOST_JOIN(_expected_,V) = EXPR; \
if (!  BOOST_JOIN(_expected_,V)) 
  return make_unexpected(BOOST_JOIN(_expected_,V).error()); \
auto V =*BOOST_JOIN(_expected_,V)
\end{lstlisting}

\begin{lstlisting}
expected<exception_ptr,int> f2(int i, int j, int k) {
  EXPECT(s1, safe_divide(i, k) );
  EXPECT(s2, safe_divide(j, k));
  return s1 + s2;
}
\end{lstlisting}

\end{frame}
%%%%%%%%%%%%%%%%%%%%%%%%%%%%
\subsection{getIntRange }
%%%%%%%%%%%%

\begin{frame}[fragile]
\frametitle{getIntRange - exception based}

Given
\begin{lstlisting}
  int getInt(istream_range& r);
  void matchesString(std::string, istream_range& r);
\end{lstlisting}

\begin{lstlisting}
pair<int,int> getIntRange(istream_range& r) 
{
    auto f = getInt(r);
             matchesString("..", r);
    auto l = getInt(r);       
    return make_pair(f, l);
}
\end{lstlisting}

\end{frame}
%%%%%%%%%%%%%%%%%%%%%%%%%%%%
\begin{frame}[fragile]
\frametitle{getIntRange - expect based}

Given
\begin{lstlisting}
  expected<errc, int> getInt(istream_range& r);
  expected<errc, void> matchesString(istream_range& r);
\end{lstlisting}

\begin{lstlisting}
expected<errc, pair<int, int>> getIntRange(istream_range& r) 
{
  return 
    auto  f <- getInt(r) : 
               matchedString("..", r) : 
    auto  l <- getInt(r) :
    std::make_pair(f, l);
}
\end{lstlisting}

\end{frame}
%%%%%%%%%%%%%%%%%%%%%%%%%%%%
\section{Behind the scenes}
%%%%%%%%%%%%%%%%%%%%%%%%%%%%
\subsection{Unexpected Type}
%%%%%%%%%%%%%%%%%%%%%%%%%%%%
\begin{frame}[fragile]
\frametitle{class unexpected\_type<E>}

Between explicit and implicit conversion.

Building an explicit type that is implicitly convertible to another type.

E.g. \cpp{imaginary<T>} is explicitly convertible from a \cpp{T} and \cpp{complex<T>} could be implicitly convertible from \cpp{imaginary<T>}. 

\begin{lstlisting}
template <class E=std::exception_ptr>
class unexpected_type {
public:
    unexpected_type() = delete;
    constexpr explicit unexpected_type(E e) 
        : error_(e) {}   
    constexpr E value() const  { return error_; }                              
}; 
\end{lstlisting}

\end{frame}
%%%%%%%%%%%%%%%%%%%%%%%%%%%%
\begin{frame}[fragile]
\frametitle{unexpected factories}

\begin{lstlisting}
template <class E>
constexpr unexpected_type<decay_t<E>> make_unexpected(E v)  {
    return unexpected_type<decay_t<E>> (ex);
}

unexpected_type<> 
  make_unexpected_from_current_exception() {
    return unexpected_type<> (std::current_exception());
}

template <class T, class E>
constexpr 
unexpected_type<E> make_unexpected(expected<E, T> v)  {
    return unexpected_type<E>(v.error());
}

\end{lstlisting}
\end{frame}
%%%%%%%%%%%%%%%%%%%%%%%%%%%%
\subsection{Expected Type}
%%%%%%%%%%%%%%%%%%%%%%%%%%%%
%%%%%%%%%%%%%%%%%%%%%%%%%%%%
\subsubsection{Conversions}
%%%%%%%%%%%%%%%%%%%%%%%%%%%%
\begin{frame}[fragile]
\frametitle{\cpp{expected<E, T>} conversions from/to \cpp{E} and \cpp{unexpected_type<E>}}

\begin{itemize}
\item \cpp{expected<E, T>} is not convertible from \cpp{E} .
\item \cpp{expected<E, T>} is implicitly convertible from \cpp{unexpected_type<E>} .
\item The opposite is not true. Needs explicit function \cpp{get_unexpected()}.
\end{itemize}

\begin{lstlisting}
template <class E, class T>
class expected {
  // ...
  constexpr expected(unexpected_type<E>&& e);
  expected&  operator=(unexpected_type<E>&& e);

  // ...  
  constexpr E const& error() const& noexcept;
  constexpr E& error() & noexcept;
  constexpr E&& error() && noexcept;
  constexpr unexpected_type<E> get_unexpected() const&;
  constexpr unexpected_type<E> get_unexpected() &&;
};
\end{lstlisting}

\end{frame}
%%%%%%%%%%%%%%%%%%%%%%%%%%%%
\begin{frame}[fragile]
\frametitle{\cpp{expected<exception_ptr, T>} conversions from/to \cpp{unexpected_type<E>}}

\begin{itemize}
\item \cpp{expected<exception_ptr, T>} is implicitly convertible from \cpp{unexpected_type<E>}  for any \cpp{E}.
\end{itemize}

\begin{lstlisting}
template <class T>
class expected<exception_ptr, T> {
  // ...
  constexpr expected(unexpected_type<exception_ptr>&& e);
  expected&  operator=(unexpected_type<exception_ptr>&& e);
  template <class E>
  constexpr expected(unexpected_type<E>&& e);
  template <class E>
  expected&  operator=(unexpected_type<E>&& e);

  // ...  
};
\end{lstlisting}

\end{frame}
%%%%%%%%%%%%%%%%%%%%%%%%%%%%
\begin{frame}[fragile]
\frametitle{\cpp{expected<E, T>} conversions from/to \cpp{T}}

\begin{itemize}
\item \cpp{expected<E, T>} is implicitly convertible from \cpp{T} .
\item The opposite is not true. Needs explicit function \cpp{value()}.
\end{itemize}

\begin{lstlisting}
template <class E, class T>
class expected {
  // ...
  constexpr expected(T const& v);
  constexpr expected(T&& v);
  expected&  operator=(T&& v);
  
  constexpr T const& value() const&;
  constexpr T& value() &;
  constexpr T&& value() &&;
  
};
\end{lstlisting}

\end{frame}
%%%%%%%%%%%%%%%%%%%%%%%%%%%%
\subsubsection{Constructors}
%%%%%%%%%%%%%%%%%%%%%%%%%%%%
\begin{frame}[fragile]
\frametitle{Default constructor}
While \cpp{expected<exception_ptr, T>} is default constructible, there is no exception stored in.

Alternatives: 

\begin{itemize}
\item value() is undefined behavior if there is no exception stored or 
\item it throws a specific exception if there is no exception stored.
\end{itemize}

\end{frame}
%%%%%%%%%%%%%%%%%%%%%%%%%%%%
\subsubsection{Observers}
%%%%%%%%%%%%%%%%%%%%%%%%%%%%
\begin{frame}[fragile]
\frametitle{\cpp{pointer-like}}

\begin{lstlisting}
template <class E, class T>
class expected {
  // ...
  constexpr explicit operator bool() const noexcept;

  constexpr T const& operator*() const& noexcept;
  constexpr T& operator*() & noexcept; 
  constexpr T&& operator*() && noexcept; 

  constexpr T const  * operator->() const noexcept;
  constexpr T* operator*() noexcept; 
  
};
\end{lstlisting}

\end{frame}
%%%%%%%%%%%%%%%%%%%%%%%%%%%%
\subsubsection{Monadic functions}
%%%%%%%%%%%%%%%%%%%%%%%%%%%%
\begin{frame}[fragile]
\frametitle{Monadic functions}


\begin{tabular}{|c|c|c|c|c|}
\hline
                    & Object & From & To & Result  \\
\hline
\textbf{fmap} & \cpp{M<E,T>} & \cpp{T} & \cpp{U} & \cpp{M<E,U>} \\
\hline
\textbf{fmap} & \cpp{M<E,T>} & \cpp{T} & \cpp{M<E,U>} & \cpp{M<E,M<E,U>>} \\
\hline
\textbf{mbind} & \cpp{M<E,T>} & \cpp{T} & \cpp{U} & \cpp{M<E,U>} \\
\hline
\textbf{mbind} & \cpp{M<E,T>} & \cpp{T} & \cpp{M<E,U>} & \cpp{M<E,U>} \\
\hline
\textbf{catch\_error} & \cpp{M<E,T>} & \cpp{E} & \cpp{T} & \cpp{M<E,T>} \\
\hline
\textbf{catch\_error} & \cpp{M<E,T>} & \cpp{E} & \cpp{M<E,T>} & \cpp{M<E,T>} \\
\hline
\textbf{then} & \cpp{M<E,T>} & \cpp{M<E,T>} & \cpp{U} & \cpp{M<E,U>} \\
\hline
\textbf{then} & \cpp{M<E,T>} & \cpp{M<E,T>} & \cpp{M<E,U>} & \cpp{M<E,U>} \\
\hline
\end{tabular}

\end{frame}

%%%%%%%%%%%%%%%%%%%%%%%%%%%%
\begin{frame}[fragile]
\frametitle{\cpp{fmap} member}

\begin{lstlisting}
  template <typename F>
  constexpr expected<E, result_of_t<F(T)>>
  expected<E,T>::fmap(F&& f,
    REQUIRES( ! is_void_v<result_of_t<F(T)>> )
  ) const 
  {
    typedef expected<E, result_of_t<F(T)>> result_type;
    if (*this) return result_type(f(**this));
    else       return get_unexpected();
  }
\end{lstlisting}

\end{frame}
%%%%%%%%%%%%%%%%%%%%%%%%%%%%
\begin{frame}[fragile]
\frametitle{\cpp{mbind} member}

\begin{lstlisting}
  template <typename F>
  constexpr expected<E, result_of_t<F(T)>>
  expected<E,T>::mbind(F&& f,
    REQUIRES( ! is_void_v<result_of_t<F(T)>> )
  ) const 
  {
    typedef expected<E, result_of_t<F(T)>> result_type;
    if (*this) return f(**this);
    else       return get_unexpected();
  }
\end{lstlisting}

\end{frame}
%%%%%%%%%%%%%%%%%%%%%%%%%%%%
\begin{frame}[fragile]
\frametitle{\cpp{catch_error} member}

\begin{lstlisting}
template <typename F>
constexpr expected<E,T>
expected<E,T>::catch_error(F&& f,
  REQUIRES( is_same_v<result_of_t<F(E)>, expected<E,T>> ) 
) const
{
  if ( ! *this ) return f(error());
  else           return *this;
}
\end{lstlisting}

\end{frame}
%%%%%%%%%%%%%%%%%%%%%%%%%%%%
\begin{frame}[fragile]
\frametitle{\cpp{then}}

\begin{lstlisting}
template <typename F>
constexpr result_of_t<F(T)>
expected<E,T>::then(F&& f,
  REQUIRES( is_expected_v<result_of_t<F(expected<E,T>)> > ) 
) 
{
  f(move(*this));
}
\end{lstlisting}

\end{frame}
%%%%%%%%%%%%%%%%%%%%%%%%%%%%
\subsection{Operators}
%%%%%%%%%%%%%%%%%%%%%%%%%%%%
\begin{frame}[fragile]
\frametitle{Relational operators}

\begin{itemize}
\item unexpected values are always less than expected ones.
\item How unexpected values compare? 
\item \cpp{std::exception_ptr} doesn't compare.
\item Do we want \cpp{expected<exception_ptr,T>} be comparable?
\item compare as \cpp{optional<T>}?
\item If yes, all the \cpp{unexpected_type<exception_ptr>} compare equals.
\item This seems counterintuitive, as the observable behavior is different.

\item \cpp{expected<E,T>} is comparable if \cpp{unexpected_type<E>} and \cpp{T} are comparable.
\item \cpp{unexpected_type<E>} is comparable  if \cpp{E} is comparable and
\item \cpp{unexpected_type<std::exception_ptr>} is not comparable.
\end{itemize}

\end{frame}
%%%%%%%%%%%%%%%%%%%%%%%%%%%%
\begin{frame}[fragile]
\frametitle{Relational operators}

\begin{lstlisting}
template <class T, class E>
constexpr bool 
operator<(const expected<E,T>& x, const expected<E,T>& y)
{
  return (x) 
    ? (y) ? *x < *y : false
    : (y) ? true : x.get_unexpected()<y.get_unexpected();
}

template <class T, class E>
constexpr bool 
operator==(const expected<E,T>& x, const expected<E,T>& y)
{
  return (x)
    ? (y) ? *x == *y : false
    : (y) ? false : x.get_unexpected()==y.get_unexpected();
}
\end{lstlisting}

\end{frame}
%%%%%%%%%%%%%%%%%%%%%%%%%%%%
\subsection{Factories}
%%%%%%%%%%%%%%%%%%%%%%%%%%%%
\begin{frame}[fragile]
\frametitle{expected factories}

\begin{lstlisting}
template<typename T>
constexpr expected<std::exception_ptr, decay_t<T> > 
  make_expected(T&& v ) ;
expected<std::exception_ptr, void> make_expected();
template <typename T, typename E>
expected<std::exception_ptr, T> 
  make_expected_from_exception(E e)
template <typename T, typename E>
constexpr expected<decay_t<E>, T> 
  make_expected_from_error(E e);
\end{lstlisting}

\begin{lstlisting}
auto e1 = make_expected(2); // expected<exception_ptr,int>
auto e2 = make_expected_from_exception<int>(bad_alloc()); 
		// expected<exception_ptr,int>
auto e3 = make_expected_from_error<int>(errc::invalid); 
		// expected<errc,int>
\end{lstlisting}
\end{frame}
%%%%%%%%%%%%%%%%%%%%%%%%%%%%
\begin{frame}[fragile]
\frametitle{expected<E> as a type constructor}

\begin{lstlisting}
auto e1 = make_expected<errc>(2); // compile fails
auto e1 = expected<errc,int>(2);  // int is redondant
auto e2 = expected<errc>::make(2);  // no redondant
\end{lstlisting}

\begin{lstlisting}
template <typename E=std::exception_ptr, class T=holder>
class expected;
template <typename E>
class expected<E,holder> {
public:
  template <class T> using type = expected<E,T>;
  template <class T> expected<E,T> make(T&& v) {
    return expected<E,T>(std::forward(v));
  }
};
\end{lstlisting}
\end{frame}
%%%%%%%%%%%%%%%%%%%%%%%%%%%%
\begin{frame}[fragile]
\frametitle{Differences between \cpp{expected<E, T>} and \cpp{expected<exception_ptr, T>}}

\begin{tabular}{|c|c|c|}
\hline
                    & \cpp{expected<E, T>} & \cpp{expected<exception_ptr, T>}  \\
\hline
\textbf{never-empty warranty} & if \cpp{E} & almost yes \\
\hline
\textbf{relational operators} & if \cpp{E} and \cpp{T} & no \\
\hline
\textbf{hash} & if \cpp{E} and \cpp{T} & no \\
\hline
\textbf{has\_exception} & no & yes \\
\hline
\textbf{catch\_exception} & no & yes \\
\hline
\end{tabular}

\begin{itemize}
  \item Should \cpp{expected<E,T>} and \cpp{expected<exception_ptr,T>} be represented by two different classes?
\end{itemize}

\end{frame}
%%%%%%%%%%%%%%%%%%%%%%%%%%%%
\subsection{Related types}
%%%%%%%%%%%%%%%%%%%%%%%%%%%%
\begin{frame}[fragile]
\frametitle{\cpp{boost::variant<unexpected_type<E>,T>} Comparison}

\begin{tabular}{|c|c|c|}
\hline
                    & \textbf{variant} & \textbf{expected}  \\
\hline
\textbf{never-empty warranty} & yes & almost yes \\
\hline
\textbf{factories} & no & yes  \\
\hline
\textbf{value\_type} & no & yes  \\
\hline
\textbf{default constructor} & yes (if E is ) & yes (if E is )  \\
\hline
\textbf{observers} & get<T>/get<E> & operator*/value/error   \\
\hline
\textbf{visitation} & apply\_visitor & fmap/mbind/catch\_error  \\
\hline
\end{tabular}
\end{frame}

%%%%%%%%%%%%%%%%%%%%%%%%%%%%
\begin{frame}[fragile]
\frametitle{\cpp{std::future<T>} and \cpp{expected<exception_ptr,T>} Comparison}

\begin{tabular}{|c|c|c|}
\hline
                    & \textbf{future} & \textbf{expected}  \\
\hline
\textbf{specific null value} & no & no \\
\hline
\textbf{relational operators} & no & depends \\
\hline
\textbf{factories} & \cpp{make_ready_future} & \cpp{make_expected}  \\
\hline
\textbf{factories} & no & \cpp{make_unexpected}  \\
\hline
\textbf{emplace} & no & yes \\
\hline
\textbf{value\_type} & no & yes  \\
\hline
\textbf{default constructor} & yes(invalid) & yes (\cpp{E()} )  \\
\hline
\textbf{state} & \cpp{valid} / \cpp{ready} & \cpp{operator bool}   \\
\hline
\textbf{observers} & \cpp{get} /\cpp{ wait} & \cpp{operator*}/\cpp{value}/\cpp{error}   \\
\hline
\textbf{visitation} & \cpp{then} & \cpp{fmap}/\cpp{mbind}/\cpp{catch_error}  \\
\hline
\textbf{grouping}  &  \cpp{when_all}/\cpp{when_any} &  n.a. \\
\hline
\end{tabular}
\end{frame}

%%%%%%%%%%%%%%%%%%%%%%%%%%%%
\begin{frame}[fragile]
\frametitle{expected/future differences}

\begin{itemize}
\item Should we add a \cpp{make<future<>>} as \cpp{make<expected<E>>}?
\item Should we make \cpp{future<T>} be implicitly constructible from \cpp{unexpected_type<exception_ptr>>}?
\item Should we add emplace functions for \cpp{future<T>}?
\item Should we add nested \cpp{value_type} to \cpp{future<T>}?
\item Should we add  \cpp{valid()} and \cpp{ready()} functions to \cpp{expected<E,T>} that return always true?
\item Should we add  \cpp{operator bool} to \cpp{future<T>}?
\item Should we add  \cpp{fmap()}/\cpp{mbind()}/\cpp{catch_error()}  functions to \cpp{future<T>}?
\item Should we add  \cpp{value()} to \cpp{future<T>}?
\item Should we add  \cpp{error()} to \cpp{future<T>}?
\end{itemize}

\end{frame}
%%%%%%%%%%%%%%%%%%%%%%%%%%%%
\begin{frame}[fragile]
\frametitle{Conversions from/to ready std::experimental::future<T> }

\begin{lstlisting}
template <class T>
expected<exception_ptr,T> make_expected(future<T>&& f) {
  assert (f.ready() && "future not ready");
  try {
    return f.get();
  } catch (...) {
    return make_unexpected_from_exception();
  }
}
\end{lstlisting}

\begin{lstlisting}
template <class T>
std::future<T> 
make_ready_future(expected<exception_ptr, T>&& e) {
  if (e) return make_ready_future(*e);
  else return make_unexpected_future<T>(e.error()); 
}
\end{lstlisting}
\end{frame}

%%%%%%%%%%%%%%%%%%%%%%%%%%%%
\begin{frame}[fragile]
\frametitle{\cpp{make_unexpected_future} }

\begin{lstlisting}
template <class T, class E>
std::future<T> make_unexpected_future(E e)  {
  std::promise<T> p;
  std::future<T> f = p.get_future();
  p.set_exception(std::make_exception_ptr(e));
  return std::move(f);
}
\end{lstlisting}

\end{frame}

%%%%%%%%%%%%%%%%%%%%%%%%%%%%
\begin{frame}[fragile]
\frametitle{\cpp{make_expected} alternative implementation }

\begin{itemize}
\item If  \cpp{future<T>} stores the exception on a  \cpp{exception_ptr}, could define this function as a friend function.   
\end{itemize}

\begin{lstlisting}
template <class T>
expected<exception_ptr, T> make_expected(future<T>&& f) {
  assert (f.ready() && "future not ready");
  lock_guard<mutex> lk(f.get_mutex());
  if (f.has_value(lk)) return f.get(lk);
  else return make_unexpected(f.get_exception_ptr(lk));
}
\end{lstlisting}

\begin{itemize}
\item If \cpp{future<T>} stores a \cpp{expected<exception_ptr,T>}.
\end{itemize}

\begin{lstlisting}
template <class T>
expected<exception_ptr, T> make_expected(future<T>&& f) {
  return expected<exception_ptr, T>(f);
}
\end{lstlisting}

\end{frame}
%%%%%%%%%%%%%%%%%%%%%%%%%%%%
\begin{frame}[fragile]
\frametitle{std::experimental::optional<T> Comparison}

\begin{itemize}
  \item \cpp{expected<E, T>} is as an \cpp{std::experimental::optiona<T>l} that collapse all the values of \cpp{E} to \cpp{nullopt}.
  \item \cpp{expected<nullopt_t, T>} should behave as much as possible as \cpp{optional<T>}.
\end{itemize}
  
\begin{tabular}{|c|c|c|}
\hline
                    & \textbf{optional} & \textbf{expected}  \\
\hline
\textbf{specific null value} & yes & no \\
\hline
\textbf{relational operators} & yes & depends \\
\hline
\textbf{swap} & yes & yes \\
\hline
\textbf{factories} & \cpp{make_optional} & \cpp{make_expected}  \\
\hline
\textbf{value\_type} & no & yes  \\
\hline
\textbf{default constructor} & yes(\cpp{nullopt})  & yes (\cpp{E()} )  \\
\hline
\textbf{observers} & \cpp{value}  & \cpp{value} / \cpp{error}   \\
\hline
\textbf{unwrap} & no  & yes   \\
\hline
\textbf{visitation} & no & \cpp{fmap}/\cpp{mbind}/\cpp{catch_error}  \\
\hline
\end{tabular}
\end{frame}
%%%%%%%%%%%%%%%%%%%%%%%%%%%%
\begin{frame}[fragile]
\frametitle{expected/optional Differences }

\begin{itemize}
\item Should we add a \cpp{make<optional<>>} as \cpp{make<expected<E>>}?
\item Should we make \cpp{expected<E,T>} be implicitly constructible from \cpp{E>}?
\item Should we add nested \cpp{value_type} to \cpp{optional<T>}?
\item Should we add  \cpp{fmap()}/\cpp{mbind()}/\cpp{catch_error()}  functions to \cpp{optional<T>}?
\item Should we add  \cpp{error()} to \cpp{optional<T>}?
\item Should we add  \cpp{unwrap()} to \cpp{optional<T>}?
\end{itemize}

\end{frame}

%%%%%%%%%%%%%%%%%%%%%%%%%%%%
\begin{frame}[fragile]
\frametitle{std::experimental::optional<T> Conversions}

\begin{lstlisting}
template <class T>
optional<T> make_optional(expected<E, T> v) {
  if (v) return make_optional(*v);
  else nullopt;
}
\end{lstlisting}

\begin{lstlisting}
struct conversion_from_nullopt {};
template <class T>
expected<exception_ptr, T> make_expected(optional<T> v) {
  if (v) return make_expected(*v);
  else make_unexpected(conversion_from_nullopt());
}
\end{lstlisting}
\end{frame}

\section{Open Points, Future Work and Conclusions}
%%%%%%%%%%%%%%%%%%%%%%%%%%%%
\begin{frame}
\frametitle{Open Points ...}

\begin{itemize}
  \item Should the \cpp{operator==} collapse all the unexpected values?
  \item Should \cpp{expected<errc,T>} and \cpp{expected<exception_ptr,T>} be represented by two different classes?
  \item Should \cpp{expected<E,T>} throw \cpp{E} instead of \cpp{bad_expected_access}?
  \item Should \cpp{expected<E,T>} be convertible from \cpp{E} when it is not convertible to \cpp{T}?
  \item Should  \cpp{fmap()}/\cpp{mbind()}/\cpp{catch_error()} expect that the continuation doesn't throws?
  \item Should we have operators for  \cpp{fmap()}/\cpp{mbind()}/\cpp{catch_error()} (\cpp{^}/\cpp{&}/\cpp{|})?

\end{itemize}

\end{frame}
%%%%%%%%%%%%%%%%%%%%%%%%%%%%%%%%%%%%
\begin{frame}[fragile]
\frametitle{Infix operators}

\begin{lstlisting}
auto e2 = e1 ^ f1 ^ f2;
\end{lstlisting}

\begin{lstlisting}
auto e2 = e1 ^fmap^ f1 ^fmap^ f2;
\end{lstlisting}

\begin{lstlisting}
return ( e1 <- f1 : e2 <- f2 : e1+e2 ) | rec;
\end{lstlisting}

\begin{lstlisting}
return ( e1 <- f1 : e2 <- f2 : e1+e2 ) ^catch_error^ rec;
\end{lstlisting}


\end{frame}


%%%%%%%%%%%%%%%%%%%%%%%%%%%%
\begin{frame}
\frametitle{Future Work ...}

\begin{itemize}
  \item Allocator support for expected.
  \item Define a common visitation interface for \cpp{any}, \cpp{variant<E1, ..., En>}, \cpp{exception_ptr}.
  \item Define an Error concept that would make \cpp{expected<Error,T>} more generic.
\end{itemize}

\end{frame}
%%%%%%%%%%%%%%%%%%%%%%%%%%%%
\begin{frame}
\frametitle{Conclusions}

\begin{itemize}
  \item \cpp{expected<E, T>} monadic functions are useful tools, 
  \item that allows to combine functions that return \cpp{expected<E, T>} but,
  \item it would be much easier to use it with a specific language do-expression.
\end{itemize}

\begin{itemize}
  \item \cpp{expected<E, T>}, \cpp{future<T>} and \cpp{optional<T>} share a lot of things but have some differences.
  \item Defining a common interface for the functions that have the same behavior allows us to define generic algorithms on top of these concepts.
  \item Having a monadic common interface would be one step towards this goal.
\end{itemize}

\end{frame}
%%%%%%%%%%%%%%%%%%%%%%%%%%%%
\begin{frame}
\begin{center}
\Large{Thanks for your attention!}
\end{center}
\begin{center}
\Large{Questions?}
\end{center}
\end{frame}



%%%%%%%%%%%%%%%%%%%%%%%%%%%%%%%%%%%%


\end{document}
